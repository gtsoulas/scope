%%% response.tex ---
%%
%% Filename: response.tex
%% Description:
%% Author: George Tsoulas
%% Maintainer:
%% Created: Thu Dec  1 21:59:20 2016 (+0000)
%% Version:
%% Package-Requires: ()
%% Last-Updated:
%%           By:
%%     Update #: 0
%% URL:
%% Doc URL:
%% Keywords:
%% Compatibility:
%%
%%%%%%%%%%%%%%%%%%%%%%%%%%%%%%%%%%%%%%%%%%%%%%%%%%%%%%%%%%%%%%%%%%%%%%
%%
%%% Commentary:
%%
%%
%%
%%%%%%%%%%%%%%%%%%%%%%%%%%%%%%%%%%%%%%%%%%%%%%%%%%%%%%%%%%%%%%%%%%%%%%
%%
%%% Change Log:
%%
%%
%%%%%%%%%%%%%%%%%%%%%%%%%%%%%%%%%%%%%%%%%%%%%%%%%%%%%%%%%%%%%%%%%%%%%%
%%
%% This program is free software; you can redistribute it and/or
%% modify it under the terms of the GNU General Public License as
%% published by the Free Software Foundation; either version 3, or
%% (at your option) any later version.
%%
%% This program is distributed in the hope that it will be useful,
%% but WITHOUT ANY WARRANTY; without even the implied warranty of
%% MERCHANTABILITY or FITNESS FOR A PARTICULAR PURPOSE.  See the GNU
%% General Public License for more details.
%%
%% You should have received a copy of the GNU General Public License
%% along with this program; see the file COPYING.  If not, write to
%% the Free Software Foundation, Inc., 51 Franklin Street, Fifth
%% Floor, Boston, MA 02110-1301, USA.
%%
%%%%%%%%%%%%%%%%%%%%%%%%%%%%%%%%%%%%%%%%%%%%%%%%%%%%%%%%%%%%%%%%%%%%%%
%%
%%% Code:
\documentclass[11pt]{article}

\usepackage{linguex}

\begin{document}
\title{Response to reviewer's comments}
\author{George Tsoulas \& Norman Yeo}
\date{\today}

\maketitle


The reviewer's insightful and helpful comments have allowed us to clarify a number of issues in the paper and have led to several major modifications in the presentation and argument.  We are pleased that we have been able to incorporate the vast majority of the comments that we received.  In the very few cases where we have not been able to do so is in cases where taking into account the reviewer's commentary would require inordinate lengthening of the paper.  More precisely:

\paragraph{Reviewer 1 (Starting: ``This is a well written paper'')}  We have followed the reviewer's suggestions and incorporated earlier discussions of the idea of differential spell-out.  We have limited ourselves to the areas where this was most relevant to the concerns of the paper.  the reviewer raises the issue of the connection that we make between the analysis of WH and the Beghelli and Stowell system for scope and suggests that it may be problematic in the framework of the paper that choice function  indefinites and Q-particles do not behave alike (especially that Q-particles do not behave as existential quantifiers).  This is correct but this is only because Q-particles are \textbf{not} existential quantifiers.  Although we are not taking an explicit stance, they are usually considered to be variables over choice functions which can be bound either by existential closure or an interrogative C.  If anything this behaviour supports our proposal since it underlines a parallel between choice function indefinites which can be ``bound'' at wide or narrow scope levels by closure operations (this is because the closure operation is not restricted).
the reviewer also asks why, under this account QR would be clause bound unlike Wh movement.  We have not incorporated a discussion of this point in the paper but the answer is rather obvious under our assumptions.  given that the effects of QR are subsumed under AGREE between a functional head and a DP it follows that QR will be restricted by the same locality constraints as AGREE.

\paragraph{Reviewer 2 (Starting ``For example''}  The reviewer makes a number of small points that have been dealt with.  One point that we have not discussed is the idea that there e is a problem because the scope of indefinites seems to be insensitive to islands.  This is a vast topic but the semantic literature contains a number of proposals stemming from Kamp's and Heim's work that is aimed to account for this.  Specifically Abusch(1994) shows that the behaviour of indefinites is different from that of quantifiers is many respects.  As a result, any of these proposal would be appropriate here.  Discussing this topic in the present context would really take us too far afield.
The second comment regards the status of LF movement.  We have discussed this in more detail in the paper and show that the notion of ``LF-movement'' is in itself unformulable under current theoretical assumptions and even if it was it does not produce what one would like it to.

\paragraph{Reviewer 3.  (Starting ``The paper argues'')} This was by far the most detailed of reviews, and we thank the reviewer for such careful consideration. The reviewer asks what the novel/difference of our proposal with respect to the Q-theory and we have done is to argue for an SSO approach to various copies of the QP and more significantly, discontinuous spell-out of different material in the two copies to account for discontinuous structures. This voids the need for LF-movement that Cable assumes for Sinhala (as the review points out as an objection). While it is true that our solution also resorts to interface effects (LF vs. PF), we have essentially established an underlying uniformity of syntax across the typological range of wh-constructions, without positing the need for Q-adjunction vs. Q-complementation. The review further points out the issue of Japanese, suggesting that \textit{ittai} suggests that Q moves on its own to the exclusion of the wh-phrase. We present a relatively complete set of facts, and specifically an extra set of data that Hagstrom himself uses to argue for Q-migration, where Q is always merged locally to a wh-word before \textbf{overtly} moving, even out of an island if necessary. We argue that such overt movement (Q-migration or otherwise), cannot preclude the pied-piping of the wh-phrase with it and leaving the relevant copies. Discontinuous spell-out takes care of the rest. The next point address Japanese and Vata, where the reviewer notes that if Q and wh do not move together, then the semantics for the wh-word cannot obtain. As mentioned above, we argue that the QP always moves together, what we have is discontinuous spellout. Given that Q is a variable over choice functions, multiple copies of [Q wh] will all be bound by either existential closure or more relevantly interrogative C. This results in the desired wh-indefinite reading in all copies, while avoiding the issue of copy non-uniformity of operator--variable chains. Interrogative C then provides propositional set formation, as usual. The final two points the reviewer raises has to do with multiple wh-questions, superiority effects and multiple wh-movement. These are rather large issues in themselves and warrant a separate paper. For reasons of time and space, it was simply not possible to include these facts at these time but we will be certain to turn our attention to these issues for future research.




\end{document}



%%%%%%%%%%%%%%%%%%%%%%%%%%%%%%%%%%%%%%%%%%%%%%%%%%%%%%%%%%%%%%%%%%%%%%
%%% response.tex ends here
