%%% response-reviewer1.tex ---
%%
%% Filename: response-reviewer1.tex
%% Description:
%% Author: George Tsoulas
%% Maintainer:
%% Created: Fri Dec  2 10:58:03 2016 (+0000)
%% Version:
%% Package-Requires: ()
%% Last-Updated:
%%           By:
%%     Update #: 0
%% URL:
%% Doc URL:
%% Keywords:
%% Compatibility:
%%
%%%%%%%%%%%%%%%%%%%%%%%%%%%%%%%%%%%%%%%%%%%%%%%%%%%%%%%%%%%%%%%%%%%%%%
%%
%%% Commentary:
%%
%%
%%
%%%%%%%%%%%%%%%%%%%%%%%%%%%%%%%%%%%%%%%%%%%%%%%%%%%%%%%%%%%%%%%%%%%%%%
%%
%%% Change Log:
%%
%%
%%%%%%%%%%%%%%%%%%%%%%%%%%%%%%%%%%%%%%%%%%%%%%%%%%%%%%%%%%%%%%%%%%%%%%
%%
%% This program is free software; you can redistribute it and/or
%% modify it under the terms of the GNU General Public License as
%% published by the Free Software Foundation; either version 3, or
%% (at your option) any later version.
%%
%% This program is distributed in the hope that it will be useful,
%% but WITHOUT ANY WARRANTY; without even the implied warranty of
%% MERCHANTABILITY or FITNESS FOR A PARTICULAR PURPOSE.  See the GNU
%% General Public License for more details.
%%
%% You should have received a copy of the GNU General Public License
%% along with this program; see the file COPYING.  If not, write to
%% the Free Software Foundation, Inc., 51 Franklin Street, Fifth
%% Floor, Boston, MA 02110-1301, USA.
%%
%%%%%%%%%%%%%%%%%%%%%%%%%%%%%%%%%%%%%%%%%%%%%%%%%%%%%%%%%%%%%%%%%%%%%%
%%
%%% Code:

\documentclass[11pt]{article}

\usepackage{linguex}
\usepackage{todonotes}
\usepackage{natbib}
\bibliographystyle{chicago}

\newcommand{\citeposs}[1]{\citeauthor{#1}'s (\citeyear{#1})}



\begin{document}
\title{Response to the second reviewer's comments}
\author{this is the review starting \textit{For example\ldots}}
\date{December 2016}

\maketitle
We would like to begin by thanking the reviewer for his/her time and effort put into reading and commenting on the paper. Clearly, the points raised have led to a number of modifications that have led to several improvements in clarity, presentation and content.  The reviewer has raised a number of points of detail and presentation on specific places of the paper. We will not list them all here but we have taken them all into account and either have clarified the point/presentation or made the required change. In the following we will address two points that relate more to the content of the paper.

The first point is where the reviewer sees a problem in the scope of indefinites. Including a full discussion of this would, of course, take us too far afield. We should note, however, that as it seems to us, there is no real problem in this respect. The phenomenon of indefiniteness scoping out of islands has received a number of treatments in the literature.  For example, one could follow \citeposs{abusch:1994a} extension of the Kamp--Heim framework for the analysis of indefinites and analyse them accordingly. On the other hand, choice-function approaches to wide-scope for indefinites, as pioneered by \citet{reinhart:1997} and \citet{winter:1997a} will also provide not only an appropriate analysis but also, and more importantly for our concerns here, a parallel with the behaviour of Q-particles, the latter being potentially interpreted as (variables) over choice function that interact with question words to yield indefinites or questions, depending on the operator that binds them, with the former taking wide or narrow scope depending on the level where closure operations (including $\exists$-closure or choice function) applies.

The second point that the reviewer raises regards the status of LF movement. Specifically, the reviewer suggests that the copy theory of movement merely integrates LF into the syntax, and LF movement and spell-out of lower copies are notational variants of each other. We respectfully disagree with this position for the following reasons: first, given current assumptions regarding phase-based derivation, LF movement of the type that is assumed generally becomes unformulable. The difference between a theory based on the more traditional Y-model and a phase-based one is that in the latter there can be no syntactic operations on the ``LF-branch''. Of specific concern is the notion of movement under pre-copy theory frameworks that require operator--variable chains be formed -- in fact, LF movement was proposed (in Huang's (1982) seminal work for example) precisely to establish an operator--variable chain. The LF-movement that we are trying to do away with is precisely of this sort. While many assume a copy theory of movement, few actually address the issue of the asymmetry between head and tail copies.

The second reason for us is that under a phase based approach, because the operation of TRANSFER entails that the transferred material is opaque and unavailable to syntactic operations. Under this assumption, the only way there is to maintain what appears to be empirically well motivated analyses of a range of typologically very different languages is to work with the idea that the ``overt'' syntax (or rather, plainly the syntax) is richer than what the acceptance of the overt vs. covert distinction of syntactic operations may lead us to believe. Put differently, what this means is that the issues that we traditionally assume to be a result of a distinction between overt and LF-movement \textbf{requires} that they now be expressed in terms of selective spell-out, which entails some uniformity of overt syntactic operations. As we show in the revised paper, especially around the facts regarding Q-migration and islands, this is non-trivial because under SSO, regardless of which copy is spelled-out, it involves movement that is subject to the same constraints.


\bibliography{fbib}


\end{document}


%%%%%%%%%%%%%%%%%%%%%%%%%%%%%%%%%%%%%%%%%%%%%%%%%%%%%%%%%%%%%%%%%%%%%%
%%% response-reviewer1.tex ends here
