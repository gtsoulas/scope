% see http://info.semprag.org/basics for a full description of this template
\documentclass{glossa}
% possible options:
% [times] for Times font (default if no option is chosen)
% [cm] for Computer Modern font
% [lucida] for Lucida font (not freely available)
% [brill] open type font, freely downloadable for non-commercial use from http://www.brill.com/about/brill-fonts; requires xetex
% [charis] for CharisSIL font, freely downloadable from http://software.sil.org/charis/
% for the Brill and CharisSIL fonts, you have to use the XeLatex typesetting engine (not pdfLatex)
% [biblatex] for using biblatex
% [linguex] loads the linguex example package
% !! a note on the use of linguex: in glossed examples, the third line of the example (the translation) needs to be prefixed with \glt. This is to allow a first line with the name of the language and the source of the example. See example (2) in the text for an illustration.
% !! a note on the use of bibtex: for PhD dissertations to typeset correctly in the references list, the Address field needs to contain the city (for US cities in the format "Santa Cruz, CA")

%\addbibresource{sample.bib} % this is for use with biblatex; replace this by the name of your bib-file (extension .bib is required); comment out if you use natbib

\usepackage{sectsty} %control style of section headings
\allsectionsfont{\normalfont\sffamily\bfseries} %sans serif boldface in section headings
\let\B\relax %to resolve a conflict in the definition of these commands between xyling and xunicode (the latter called by fontspec, called by charis)
\let\T\relax
%\usepackage{xyling} %for trees; the use of xyling with the CharisSIL font produces poor results in the branches. This problem does not arise with the packages qtree or forest.

\usepackage{soul}
\usepackage{latexsym}
\usepackage{amsmath}
\usepackage{tipa}
\usepackage{libertine} % comment out on final compile; run XeLaTeX with Charis
\usepackage{forest} % for nice trees!
\usepackage{linguex}

\renewcommand{\firstrefdash}{} % suppress Linguex dashes between subexamples

% \pdf* commands provide metadata for the PDF output. ASCII characters only!
\pdfauthor{George Tsoulas; Norman Yeo}
\pdftitle{Scope assignment: From wh- to QR}
\pdfkeywords{Full keyword list, separated, by, commas}

% Optional short title inside square brackets, for the running headers. If no short title is given, no title appears in the headers.

\title[Scope assignment]{Scope assignment: The case of \textit{wh-}}

% Optional short author inside square brackets, for the running headers. If no short author is given, no authors print in the headers.

\author[Tsoulas \& Yeo]% short form of the author names for the running header
{%as many authors as you like, each separated by \AND.
  \spauthor{George Tsoulas\\
  \institute{University of York}\\
  \small{Heslington, York YO10 5DD, UK\\
  george.tsoulas@york.ac.uk}
  }
  \AND
  \spauthor{Norman Yeo\\
  \institute{University of York}\\
  \small{Heslington, York YO10 5DD, UK\\
  norman.yeo@york.ac.uk}
  }%
}

\begin{document}

\sffamily
\maketitle

\begin{abstract}
This is an abstract.
\end{abstract}

\begin{keywords}
  Glossa; stylesheet; latex template;stuff
\end{keywords}

%\hrulefill

\rmfamily
\section{Introduction}
The notion of scope of an operator is fundamental for semantic computation. It is also one of the few notions that seems to have a direct translation between syntax and semantics. The syntactic structure is, in terms of scope assignment, transparent to the semantic interpretation.  In other words, the scope of operators is equal to their C-command domain.  As a result, insofar as syntax is driven by the need to satisfy interface requirements and in this case requirements of the syntax-semantics interface, a prime concern of syntactic theory is to get operators to their scope positions. From a general point of view, theoretical parsimony and economy have dictated a preference for the elimination of provably superfluous operations in the derivation of structure. Under the copy theory of movement \citep{chomsky:1993}, links in a chain are copies. In fact, under current assumptions, they are not identical and independently merged at different positions, somehow identified as part of a chain. To make the maximal use of this formal representation we propose that certain cases of LF-movement\footnote{Ideally, to be maximally economical, the entire notion LF-movement should be eliminated, but our aims here are somewhat more modest.} can be eliminated from a theory of grammar and replaced by a theory of selective spell-out (SSO), schematically shown as follows:

\ex.\label{intro10}\a.XP {\dots} \textst{XP} $\qquad$ (moved construction)
    \b.\textst{XP} {\dots} XP $\qquad$ (in-situ construction)

A consequence of the schema in \ref{intro10} is that the variation between an ex-situ vs. in-situ configuration is reduced to an issue of which copy is spelled out.

The primary burden that is placed on LF-movement is that of fixing scope; that is, the movement of a scope-taking operator to a position in the structure at which it must take scope. Two of the most common phenomena to be associated with this is that of certain wh-in-situ constructions and Quantifier Raising (QR). However, the traditional notion of LF-movement is inherently incompatible with a copy theory of movement. Pre-copy theory, LF-movement was seen to be movement that leaves behind a trace that is interpreted as a variable, while the moved element is taken to be an operator that binds such a variable. For example, a wh-in-situ construction involving LF movement would be seen as follows:

\ex.\label{intro20}\a.PF: John bought what?
   \b.LF: what John bought $x$
   \b.What is $x$, such that John bought $x$?

As \cite{fox:2002} puts it, this assumes ``that traces are fairly impoverished in their representations, and as a result it conflicts with the copy theory of movement''. Under a copy theory, such traces are treated as identical copies of the same syntactic object (albeit selected only once and remerged), and that they appear in two places in the syntactic representation with PF privileging which one to pronounce (usually the higher one). In the context of wh-in-situ, this is problematic, because either the wh-phrase does not move at all, which would then require alternative mechanisms for scope fixing (e.g. unselective binding), or that the wh-phrase moves but the lower copy is privileged. Under the latter view, this results in an apparent contradiction because one is forced to say that the higher copies are both the same but different -- the higher copy must fix scope, while the lower copy has to be interpreted as the argument of the predicate that selects it.

In this paper, we argue that if we are to be serious about copy theory of movement, then we must abandon LF-movement in the traditional sense of leaving behind a trace variable as it applies to both wh-in-situ and QR. Instead, we have to adopt a notion of selective spell-out (SSO), along the lines of what \cite{bobaljik:2002} calls `LF movement':

\exg.\label{intro30}copy$_1$ {\dots} copy$_2$\\
   LF {} PF\\

In \ref{intro30}, what we have what he terms ``LF-privileging'' of the higher copy but ``PF-privileging'' of the lower copy. However, the problem with this is that it is never the case the it is only the higher copy that has interpretation at LF and all the lower copy does is to provide a pronunciation site at PF. It is, in fact, necessary that the lower copy is also interpreted at LF as the argument of the predicate that selects it. What we then have in the case of wh-in-situ and QR is actually as follows:

\exg.\label{intro40}copy$_1$ {\dots} copy$_2$\\
   LF$_1$ {} LF$_2$/PF\\

Some of the semantic information must be interpreted at LF$_1$ (e.g. scope) while the rest of the semantic information must be interpreted at LF$_2$ (e.g. argumenthood); PF can then decide which copy to pronounce. The explanatory burden, then, falls upon specifying how the LF information is ``split'' across two sites in relation to the privileging of different PF copies. For example, \cite{fox:2002} address the issue of the split LF-interpretation by proposing that the lower copy undergoes ``trace conversion'' and is interpreted as a definite description. However, since Fox is dealing with QR in ACD, the issue of the pronunciation of lower copies does not arise. \cite{bobaljik:2002} also recognises this issue and suggests that ``\dots strikethrough at LF, at least for a lower copy, does not mean deletion at LF, but should be taken to mean deletion up to (thematic) interpretability\dots ''. The aim of this paper is to (re)think how such an approach would apply in the context of a rich typology of wh-constructions (and question particles), as well as QR.

The structure of this paper is as follows: in section 2, we overview the landscape of wh-typology and highlight the parameters necessary to unify the wide range of cross-linguistic variation we find in wh-constructions; in section 3, we show how an SSO approach allows us to explain the variation by formulating a typology expressed in terms of SSO parameters; section 4 addresses the issue of QR and explores the way it can be integrated with the ideas developed with respect to wh-scope. Section 5 concludes.


\section{Wh-Movement and wh-scope}
\subsection{The independence of clause typing and wh-scope fixing}
It is well known that languages vary in terms of whether they are wh-movement or wh-in-situ languages (and in some cases in-between). The claim here is that at least some cases of this variation reflect a purely surface phenomenon; that is, some wh-in-situ languages are, in fact, wh-movement languages, and that what varies is which wh-copy is spelled-out. Let us consider first the question as to what wh-movement actually does:

\ex.\label{wh.10}\a.\label{2.10a}What$_i$ did [John buy $t_i$]?
	\b.\label{2.10b}[John bought a book].

Wh-movement, in a sense, is the syntactic and semantic equivalent of bilocation because some part of the meaning of a wh-phrase needs to be interpreted in two places at once. A common assumption among scholars is that a wh-phrase in a wh-movement langauge is an operator, i.e. it expresses existential quantification. As such, it is necessary for the wh-phrase to move to a position in the clausal periphery such that it scopes over the entire proposition including the moved subject -- this is why wh-movement involves movement to a position in the CP-layer, which is higher than the subject, which has moved to [Spec,TP]. At the same time, because the wh-phrase is also the argument of the verb, it must be interpreted as such in the base position from which it moves. Syntactic displacement, therefore, gives rise to semantic displacement.

As mentioned in the introduction, the traditional assumption is that moving the wh-phrase (either overtly or at LF) leaves a trace that is interpreted as a variable. Wh-movement is necessary in order for the moved wh-phrase to bind the trace/variable resulting in existential quantification, as shown below. The is why wh-elements are also called wh-operators: the moved wh-element is interpreted as an existential quantifier $\exists x_{wh}$ that binds the trace left by movement, which is interpreted as a variable $x_{wh}$:

\ex.\label{wh.20}$\exists x_{wh}[\dots P(x_{wh})\ \dots]$

The expression in \ref{wh.20}, however, is still not the meaning of a question. Given our theory of syntax, wh-movement also requires a syntactic trigger that is usually expressed in terms of an interrogative C: a functional head that encodes interrogative clausal force, or ``clause type'' in the sense of Cheng (1997). In semantic terms, the set formation that results in the meaning of a question comes from this interrogative C:

\ex.\label{wh.30}\a.$\lambda p\exists x_{wh}[p = P(x_{wh})]$
   \b.The set of propositions $p$ such that there exists an $x_{wh}$ (wh-phrase) that has property $P$ (the predicate)

This straightforwardly follows from approaches to question semantics as involving a set of propositions (Hamblin 1973; Karttunnen 1977). Such an approach is also compatible with well-known cases of languages that have wh-indefinites: wh-words that are interpreted existentially in declarative contexts. In other words, the lack of interrogative C precludes the availability of question meaning, resulting in a declarative involving existential quantification as in \ref{wh.20}. The upshot is that all languages, regardless of the syntactic means by which they derive a wh-construction, must converge on the question meaning expressed in \ref{wh.30} via interrogative C. Considered in the context of a copy theory of movement, a problem immediately arises because a copy theory should not allow the formation of trace variables because they are no longer syntactic primitives, i.e. material left behind after movement. This seriously undermines the notion of treating the wh-phrase as an existential ``wh-operator'' if its movement does not introduce a variable to bind.

A potential solution to this problem can be found in the introduction of a third element in question formation. In addition to interrogative C and the wh-phrase, it is becoming increasingly common to assume that in all languages, questions are formed with the help of a Q(uestion) particle, and in the early days of GB and Minimalism, it was common to assume that Q was the realisation of an interrogative C head. However, from Watanabe (1992) to Hagstrom (1998), more recently (Cable 2007, Slade 2011, Yeo 2010) analyses posit that Q is distinct from interrogative C, and that there are, in fact, three required elements in question formation: the wh-phrase, Q, and interrogative C. The combination of Q and the wh-phrase yields an indefinite when bound by interrogative C. This will be fleshed out in more detail in Section 3, but it turns out that a desirable consequence of such an approach is that the Q+wh complex is no longer a wh-operator and does not need to bind a variable. This allows us to maintain uniformity of copies across the head and tail of the chain. Before addressing the theoretical issues, let us first consider the empirical surface phenomena that an SSO approach will need to account for, keeping in mind that the ultimate aim is to explain the surface variation in terms of the spellout of different copies of the Q+wh complex.

Empirically, we find that the position of the wh-phrase and Q can vary along at least two dimensions:\footnote{There are other potential dimensions of variation that we do not discuss here. One is the headedness of the QP in languages that have clear QP constituents, e.g. Sinhala, Tlingit; another is the landing site of wh-movement, e.g. movement to Force vs. Focus positions.} first, the particle can occur in a clause-initial position, clause-final position or in a position adjacent to the wh-word; second, the wh-phrase can occur in a clause-initial (wh-movement) or base position (wh-in-situ). For expository purposes, we will postpone discussion of optional wh-movement and partial wh-movement cases to section 3.

The broad typological distribution is as follows. Wh-movement languages that have initial, wh-adjacent and final particles are exemplified by Hopi, Tlingit and Vata, respectively:


\exg.\label{wh.50}ya haki-y {\textraiseglotstop\textbari}m w{\textbari}va{\textraiseglotstop}ta \\
  Q who-\textsc{obl} you hit \\
  \trans{`Who did you hit?'}\hspace{\fill}[Hopi, \cite{jeanne:1978}]

\exg.\label{wh.60}daa s\'a k\'eet a\underline{x}\'a \\
  what Q killerwhale he.eats.it \\
  \trans{`What do killerwhales eat?'}\hspace{\fill}[Tlingit, \cite{cable:2007}]

\exg.\label{wh.70}\`al\'O$_i$ \`n n{\textvbaraccent{I}} [z\={E}$_j$ [\`a ny{\textvbaraccent{E}}-{\textroundcap{b}\textvbaraccent{O}} $t_i$ $t_j$]] y\`i l\`a\\
  who you \textsc{neg-aux} thing we gave-\textsc{rel} {} {} know Q \\
  \trans{`To whom don't you know what we have given?'}\hspace{\fill}[Vata, \cite{koopman:1984}]

Likewise for wh-in-situ languages, we find initial, wh-adjacent and final particles in Tumbuka, Sinahla and Japanese, respectively:

\exg.\label{wh.80}kasi Su\v{z}o a-ka-p\textsuperscript{h}ik-a vi\v{c}i \\
  Q Su\v{z}o \textsc{sm-pst}-cook-\textsc{fv} what \\
  \trans{`What did Su\v{z}o cook?'}\hspace{\fill}[Tumbuka, \cite{kimper:2006}]

\exg.\label{wh.90}Sunil mon{\textschwa}wa {d\textschwa} kieuwe \\
  Sunil what Q read.\textsc{pst.e} \\
  \trans{`What did Sunil read?'}\hspace{\fill}[Sinahala, \cite{slade:2011}]

\exg.\label{wh.100}John-ga nani-o nomimasita ka \\
  John-\textsc{nom} what-\textsc{acc} drank Q \\
  \trans{`What did John drink?'}\hspace{\fill}[Japanese, \cite{hagstrom:1998}]

If it is true that all questions must essentially converge on the same meaning, as shown in \ref{wh.30}, then the null hypothesis should be that all the relevant features interpretable at the interface which give rise to question meaning should be equally present in all of the above examples. Of course, other morphosyntactic properties such as case and agreement etc. can vary, but the core features responsible for question meaning (C, Q, wh, and perhaps focus) should not.

What this means for the syntax is that following the Clausal Typing Hypothesis of \cite{cheng:1997},\footnote{In this paper, we agree with Cheng in saying that clauses must be typed. We think that to the extent that there is one, clause typing is the immutable syntactic universal that relates to question formation, regardless of the syntactic framework one chooses to adopt. We do not, however, subscribe to the specific predictions that the clausal typing hypothesis makes with respect to the strict connection between wh-in-situ and the availability of question particles. See \cite{bruening:2007} for more substantial argumentation on this point.} clauses must be typed. We take this to mean only one thing, namely that that some clause must be specified as being an interrogative (as opposed to say, a declarative) by the merging of an interrogative C in the clause. The central claim here is that clause typing itself is not contingent on anything else, e.g. wh-movement or the presence of a question particle. The clearest place to find evidence of this would be take a slight detour to consider polar questions, since we know that both polar questions and wh-questions can be licensed by interrogative C and particles.\footnote{Of course, the specific features that constitute interrogative C and the question particle in polar and wh-questions may differ.} The World Atlas of Language Structures Online (WALS) contains a chapter (Ch. 116) on polar questions \citep{wals116:2013}, which shows that there are 173 languages that distinguish polar questions (only) by interrogative intonation, and 1 language (Chalcantongo Mixtec) that has no interrogative-declarative distinction. Consider the following minimal example from Kayardild (with the gloss slightly modified for consistency), an Australian Tangkic language,\footnote{According to WALS, Kayardild is listed as only using interrogative intonation, but this is inaccurate. \cite{evans:1995}, from which the Kayardild data is drawn, shows that polar questions can be formed with or without an initial question particle. This does not change our main point that other than interrogative C, no other syntactic device is necessary for clause typing.} which allows the formation of polar questions with only the use of question intonation but also optionally allows the use of a particle.

\ex.\ag.\label{wh.101a}nyingka marri-j?\\
   \textsc{2sg.nom} hear-\textsc{actual}\\
   \trans{`Do you understand?'}
   \bg.\label{wh.101b}kara nyingka marri-j?\\
   Q \textsc{2sg.nom} hear-\textsc{actual}\\
   \trans{`Can you understand?'}\hspace{\fill}\citep[364--365]{evans:1995}

Evans states: ``[polar questions] are formally identical with declaratives, except for a rising intonation contour centered on the questioned word''. Plausibly, we can assume that the ``questioned word'' in the examples here is \textit{marri-j} `hear', located at the end of the question. However, note that the question particle \textit{kara} in \ref{wh.101b} is in an initial position. One approach would be to say that the question intonation is a PF-exponent of spelling-out interrogative C, which is necessary in both \ref{wh.101a} and \ref{wh.101b}. By the Clausal Typing Hypothesis, it follows that the particle cannot be necessary. The other approach would be to say that the particle itself is interrogative C and the question intonation simply a PF-phenomenon. In which case, \ref{wh.101a} must involve a null version of \textit{kara}. Either approach reduces to requiring only interrogative C for clause typing.

A wh-question differs in that there is an additional wh-phrase that needs accounting for. However, if we extend the reasoning for polar questions to wh-questions, then we can say that like polar questions, wh-questions only require an interrogative C to clause type, but independently require something else to handle the wh-phrase -- specifically, there must be some mechanism by which the wh-phrase is made to scope over the relevant clause, for reasons stated at the beginning of this section. This is not up for debate, we think, because without a wh-scoping mechanism, the direct vs. indirect question distinction in a wh-in-situ construction cannot be derived. The question is whether clause typing and scope fixing are the same thing -- we argue that they are not.

In wh-movement languages, these two properties incidentally collapse into a single cluster of operations because interrogative C also triggers the movement of the wh-phrase into [Spec,CP], which gave rise to the (mis)understanding of wh-movement as the mechanism for clause typing. However, recall the central claim that only clause typing is universal, independent of wh-movement. In other words, clause typing and scope fixing are independent mechanisms; it just so happens that in wh-movement, they obscure each other. This point is clearer in wh-in-situ languages, because without wh-movement to obscure scope fixing, we contend with the issue of semantic displacement -- that is, the position at which a wh-phrase should be interpreted does not correspond to the syntactic position at which it appears. Even in wh-in-situ questions, clause typing must still involve the presence of interrogative C. Among the syntactic analyses of wh-in-situ, what differs is whether a particular approach treats the question particle (if it exists) as a realisation of interrogative C or not.

Therefore, if scope fixing is always required, and if wh-movement is the dominant way by which wh-scope fixing is accomplished, it is desirable to recast wh-in-situ in terms of SSO of the wh-phrase because it is a step towards unification -- clause typing and scope fixing are the same in wh-movement and wh-in-situ languages; what varies is which wh-copy is spelled-out. A reviewer notes that this criticism is unnecessary because ``the EPP is an explicit marker for the stipulation marking whether high or low copies are spelled out''. However, this is only true to the extent that there is movement to being with. If one took an unselective binding approach to wh-in-situ constructions, for example, it would be inaccurate to describe those constructions as involving the spell-out of a lower copy since there is only one copy to begin with. This is in contrast with the traditional system of expressing the availability of wh-movement in terms of the the presence of an EPP on interrogative C, because under these approaches, the traditional assumption is that wh-in-situ involves LF-movement that leaves a trace variable, which runs into the same issues we raised at the beginning of the paper. Furthermore, at LF, the purpose of wh-movement is precisely to fix scope, two things that we are arguing should be decoupled.

\subsection{The empirical profile of SSO}
Since we are assuming a copy theory of movement it is not a necessary condition for SSO to also exhibit overt morphosyntactic effects other than the pronunciation of a lower copy. Having said this, there three types of data that constitute compelling evidence for an SSO theory of wh-question formation. The first involves overt morphosyntactic effects that are linked to wh-movement but occur in a wh-in-situ construction.  This is what we find with wh-agreement phenomena in Coptic \citep{rlc:2006,reintges:2007}. The second involves what appears to be truly optional wh-movement, even in multiclausal structures, in Babine-Witsuwit'en \citep{denham:1997,denham:2000}, which we see as straightforward SSO of the wh-phrase in different CP specifiers. The third involves partial wh-movement with scope markers (e.g. in German), which will ultimately reduce to a subtype of optional wh-movement and be evidence that the particle Q must also be involved in the SSO paradigm.

To understand the analysis that Reintges proposes, we must first understand the basic phenomenon that is understood to be a diagnostic for wh-movement. To illustrate this, Reintges first considers the basic wh-agreement facts in Chamorro first, which is triggered by overt wh-movement. In Chamorro, a wh-question is formed by applying wh-movement to a declarative VSO clause, resulting in an SVO or OVS word order accompanied by corresponding case agreement on the verb:

\exg.\label{wh.110}ha-bendi si Maria i kareta\\
   \textsc{agr}-sell {} Maria the car\\
   \trans{`Maria sold the car.'}

In order to form a wh-question, wh-movement is required and a corresponding agreement marker must appear on the verb, which must agree in case with the moved wh-element:

\ex.\ag.\label{wh.120a}hayi b\textit{um}endi i kareta?\\
   who \textsc{wh}[nom].sell the car\\
   \trans{`Who sold the car?'}
   \bg.\label{wh.120b}hafa b\textit{in}inde-\textit{n\~na} si Maria?\\
   what \textsc{wh}[obj].sell-\textsc{agr} {} Maria\\
   \trans{`What did Maria sell?'}

A moved subject wh-phrase will trigger nominative case agreement on the verb, as shown by the \textit{-um-} marking on the verb in \ref{wh.120a}. A moved object wh-phrase will instead trigger objective case agreement \textit{-in-} on the verb and corresponding possessor agreement \textit{-n\~na}.

%A similar pattern is also found in relative clauses, which involve an overt relative complementiser but no overt relative pronoun:
%
%\ex.\ag.\label{wh.130a}kao un-li'i' i palao'an [ni b\textit{um}endi i kareta]?\\
%   Q \textsc{agr}-see the woman C$_{\textsc{rel}}$ \textsc{wh}[nom].sell the car\\
%   \trans{`Did you see the woman who sold the car?'}
%   \bg.\label{wh.130b}hu-fahan i kareta [ni b\textit{in}inde-\textit{n\~na} si Maria]\\
%   \textsc{agr}-buy the car C$_{\textsc{rel}}$ \textsc{wh}[obj].sell {} Maria\\
%   \trans{`I bought the car that Maria sold.'}

To sum up, Chamorro has overt wh-movement and corresponding wh-agreement, which surfaces as case agreement on the verb. What Reintges argues for is that evidence of SSO would involve behaviour that is similar to what happens in Chamorro without any overt signs of wh-movement. That is, if we have a language that exhibits overt wh-agreement with no corresponding overt wh-movement, we have evidence for spelling-out of a lower copy. We turn now to Coptic Egyptian, which appears to show precisely this kind of behaviour. In fact, in Coptic, the surfacing of wh-agreement is in complementary distribution with overt wh-movement.

Coptic Egyptian is predominantly wh-in-situ but crucially, despite the wh-phrase being in-situ, relative tense is obligatory, glossed \textsc{rel-perf} below:

\exg.\label{wh.130}{\textschwa}nt-a u \v{s}{\textopeno} {\textschwa}mmo-k pa-\v{c}oeis p-{\textschwa}rro\\
   \textsc{rel-perf} what happen to-\textsc{2sg.m} \textsc{def.sg.m.1pl}-lord \textsc{def.sg.m}-king\\
   \trans{`What happened to you, our lord and king?'}

\exg.\label{wh.140}aw{\textopeno} n-ti-sown an [t\v{s}e nt-a u \v{s}{\textopeno}pe {\textschwa}mmo-s]\\
   and \textsc{neg(-pres)-1sg}-know not C \textsc{rel-perf} what happen to-\textsc{3sg.fm}\\
   \trans{`And I don't know what happened to her.'}

The specific morphosyntactic effect that \cite{rlc:2006} claim to be wh-agreement is that of the relative tense marker (\textschwa)\textit{nt-a}. More specifically, the claim is that relative tense surfaces within the clause over which the wh-phrase takes scope: matrix scope in \ref{wh.130} and embedded scope in \ref{wh.140}. When a wh-in-situ in an embedded clause is to be interpreted with matrix scope, relative tense surfaces in the matrix clause and not the embedded clause:

\exg.\label{wh.150}eye {\textschwa}nt{\textopeno}nt{\textschwa}n e-tet{\textschwa}n-t\v{s}{\textopeno} {\textschwa}mmo-s ero-i [t\v{s}e ang nim]?\\
   Q you(\textsc{-pl}) \textsc{rel(-pres)-2pl}-say DO-\textsc{3sg:f} about-\textsc{1sg} C I who\\
   \trans{`Who are you saying of me that I (am)?'}

Note further that in \ref{wh.150}, there is the presence of an initial question particle, which serves to type the clause. In line with what we have suggested above, the data show that clause typing and scope marking are independent of each other. In Coptic, scope marking is accomplished either by wh-movement or by relative tense marking, as can be seen in the example below, where there is wh-fronting but no wh-agreement (relative tense marking):

%Consider a relative clause:
%
%\exg.\label{wh.150}u-h{\textopeno\textbeta} [ere p-nute moste mmo-f]\\
%   \textsc{indef:sg}-thing \textsc{rel(-pres)} \textsc{def:sg:m}-god hate DO-\textsc{3sg:m}\\
%   \trans{`a thing that God hates'}
%
%Like Chamorro, the formation of relative clauses in Coptic Egyptian does not make use of a relative pronoun but instead uses special wh-agreement morphology. In this case, it is the surfacing of relative tense. \cite{rlc:2006} take this to mean that like Chamorro, Coptic Egyptian relative clauses involve wh-movement of a null operator and by extension, the presence of wh-agreement suggests the presence of wh-movement.

\exg.\label{wh.160}et{\textbeta}e u ti-hmoos h{\textschwa}m pa-man\v{s}{\textopeno}pe ti-hl{\textschwa}pl{\textopeno}p?\\
   for what (\textsc{pres-})\textsc{1sg}-sit in \textsc{def.sg.m.1sg}-room (\textsc{pres-})\textsc{1sg}-be.weary\\
   \trans{`Why am I sitting in my room being weary?'}

\exg.\label{wh.170}e{\textbeta}ol t{\textopeno}n a-tet{\textschwa}-ei e-pei-ma?\\
   \textsc{pcl} where \textsc{perf-2pl}-come to-\textsc{dem:sg:m}-place\\
   \trans{`From where did you come here?'}

Here, in \ref{wh.160} we have wh-fronting of `for what' (why) and in \ref{wh.170}, we have wh-fronting of `where'. In both cases, relative tense marking does not surface. One might be tempted to argue that the distinction here is one of wh-argument (what) vs. wh-adjunct (where, why), but Reintges shows that wh-adjuncts can freely appear in-situ, which further suggests that Coptic adjunct-in-situ constructions do not involve unselective binding.\footnote{Unselective binding is known to be sensitive to the argument--adjunct asymmetry. See \cite{cheng:2009} for an overview of different approaches to unselective binding and related references.} When they do, relative tense marking surfaces:

\exg.\label{wh.180}{\textschwa}-a-k-ei e-pei-ma {\textschwa}n-a\v{s} {\textschwa}n-he?\\
   \textsc{rel-perf-2sg.m}-go to-\textsc{dem.sg.m}-place in-what of-manner\\
   \trans{`How did you get here?'}

Taken together, the data suggest that Coptic accomplishes clause typing independently of wh-phenomena; but at the same time, Coptic requires overt scope marking, which is accomplished either by overt wh-movement or by overt relative tense marking. In the latter case, relative tense marking is triggered when the wh-phrase does not overtly move. The final relevant piece of evidence that Reintges provides is to argue that the wh-in-situ constructions do indeed involve spelling-out of a lower copy as opposed to LF-movement.

The argument is as follows: if we assume a (traditional) theory of LF-movement, which establishes an operator--variable chain, then LF-movement of the operator (wh-phrase) will give rise to intervention effects if it crosses a scope bearing element. Such facts have been observed in \cite{beck:1996} and \cite{beck-kim:1997}. However, \cite{reintges:2007} argues, such intervention effects are not found in Coptic:

\exg.\label{wh.181}k-nau \v{c}e {\textschwa}nt-a-f-s{\textschwa}nt {\textschwa}m-p{\textepsilon}we t{\textepsilon}r-u {\textschwa}n-a\v{s} {\textschwa}n-he h{\textschwa}m pef-logismos\\
   \textsc{(pres).2sg.m}-see C \textsc{rel-perf-3sg.m}-establish \textsc{def.pl}-heavens all-\textsc{3pl} in-what of-manner through \textsc{def.sg.m.3sg.m}-reasoning\\
   \trans{`You see how He has established all the heavens through His reasoning.'}

In \ref{wh.181}, we observe that the wh-phrase is in-situ and is to be interpreted with embedded scope, which triggers relative tense marking in the embedded clause. Crucially, the universal quantifier \textit{t\textepsilon r-u} `all' lies along the path of wh-movement, which would, in principle, trigger an intervention effect should the wh-phrase undergo LF-movement to its scope taking position. This, Reintges concludes, is evidence that \ref{wh.181} reflects overt wh-movement rather than LF-movement, since overt movement does not trigger intervention effects. It then follows that if the wh-phrase appears in-situ without triggering intervention effects, it must be the case that the wh-phrase has overtly moved but its lower copy is pronounced.
%Conversely, when the wh-phrase does not move, relative tense surfaces:
%
%\exg.\label{wh.170}e-r-{\textbeta\textepsilon}k e-t{\textopeno}n?\\
%   \textsc{rel(-pres)-2sg:f}-go to-where\\
%   \trans{`Where are you (woman) going?'}
%
%\cite{rlc:2006} explain this distribution in terms of a similarity to a Doubly Filled Comp Filter effect \citep{chomsky-lasnik:1977} because in all cases, they argue, wh-movement takes place, but relative tense can only be pronounced when the lower copy -- but not the higher copy -- is spelled out. In sum, this is evidence for SSO of wh-phrases. They also further discuss participle agreement in relative clauses in Passamaquoddy, drawing from work done by \cite{bruening:2001}, which we do not include here for reasons for space. Instead, we turn now to Babine-Witsuwit'en, which exhibits a more straightforward effect of SSO.

We now turn to Babine-Witsuwit'en \citep{denham:1997,denham:2000}, an Athabaskan language, which appears to allow truly optional wh-movement, with no variation in morphosyntactic shape or discourse effects. The movement is also argued not to be focalisation, topicalisation or clefting, while obeying island extraction constraints as would be expected of overt movement. Compared to Coptic, the facts in Babine-Witsuwit'en are very straightforward: wh-phrases can remain in its base position or move to and stop at any Spec,CP position just as the wh-movement does not cross an island boundary. This can be clearly seen in the following:

\ex.\ag.\label{wh.190a}George [Lillian nd\"itn\"i book yik'iyelhdic] yilhn\"i?\\
   George Lillian which book \textsc{3sg}.read.\textsc{opt.3sg} \textsc{3sg}.told.\textsc{3sg}\\
   \bg.\label{wh.190b}George [nd\"itn\"i book Lillian yik'iyelhdic] yilhn\"i?\\
   George which book Lillian \textsc{3sg}.read.\textsc{opt.3sg} \textsc{3sg}.told.\textsc{3sg}\\
   \bg.\label{wh.190c}nd\"itn\"i book George [Lillian yik'iyelhdic] yilhn\"i?\\
   which book George Lillian \textsc{3sg}.read.\textsc{opt.3sg} \textsc{3sg}.told.\textsc{3sg}\\
   \trans{`Which book did George tell Lillian to read?'}

One question that will immediately arise is whether Babine-Witsuwit'en allows scrambling. It does not -- non-wh NPs do not have the freedom of moving, perhaps because of a lack of case marking on NPs:

\ex.\ag.\label{wh.200a}Lillian dus yunk\"et\\
   Lillian cat \textsc{3sg.bought.3sg}\\
   \trans{`Lillian bought a cat.'}
   \bg.*\label{wh.200b}Dus Lillian yunk\"et\\
   cat Lillian \textsc{3sg.bought.3sg}\\
   \trans{Intended meaning: `Lillian bought a cat.'}

As shown above in \ref{wh.200b}, the object cannot be fronted, and \ref{wh.200b} can only have the meaning that the cat bought Lillian. NPs can be fronted if focused, which requires the use of focus marker, which is notably absent in wh-constructions:

\exg.\label{wh.210}George 'en Lillian yunt'iy'\\
   George \textsc{foc} Lillian \textsc{3sg.likes.3sg}\\
   \trans{`It's George that Lillian likes.'}
   \trans{`It's George that likes Lillian.'}

Finally, wh-phrases are subject to island constraints. As can be seen in the examples below, extraction is impossible from sentential subjects \ref{wh.220} and coordinate structures \ref{wh.230}:

\ex.\label{wh.220}\ag.{}[George mb\"i yud\"ihye] Lillian yilhggi\"i?\\
   George who \textsc{3sg}.know.\textsc{3sg} Lillian \textsc{3sg}.surprised.\textsc{3sg}\\
   \trans{`That George knows who surprised Lillian?'}
   \bg.*mb\"i [George yud\"ihye] Lillian yilhggi\"i?\\
   who George \textsc{3sg}.know.\textsc{3sg} Lillian \textsc{3sg}.surprised.\textsc{3sg}\\

\ex.\label{wh.230}\ag.{}[George tl'ah mb\"i] hib\"in'\"e'n?\\
   George and who \textsc{2sg}.saw.\textsc{3pl}\\
   \trans{`You saw George and who?'}
   \bg.*mb\"i [George tl'ah] hib\"in'\"e'n?\\
   who George and \textsc{2sg}.saw.\textsc{3pl}\\

Denham takes this as evidence that wh-movement in Babine-Witsuwit'en, if it does occur, is overt wh-movement, and the optionality that follows is true optionality in terms of where the wh-phrase moves to. As mentioned in section 2.1, this is a classic problem of wh-scope fixing. If the wh-phrase is not pronounced at the position at which it should take scope (matrix in all the examples here), then there needs to be some mechanism by which scope can be fixed. Denham's solution to the problem involves proposing an extra projection above C, the Ty(pe)P, which is responsible for clause typing and scope. TyP is always projected where scope needs to be marked. In direct questions, TyP is projected above the matrix CP, which hosts an operator that binds the wh-phrase. Crucially, optional wh-movement is couched in terms of whether a C is projected or not, with the wh-phrase moving to wherever C is projected. If no C is projected, the wh-phrase remains in-situ, if only embedded C is projected, there is (partial) wh- movement to embedded [Spec,CP], and if matrix C is projected, there is full wh-fronting. Schematically, the solution (simplified to remove Agr projections) is formulated as follows:

\ex.\a.\label{wh.240a}wh-in-situ $=$ \ref{wh.190a}\\{}[$_{\textsc{TyP}}$ Op$_i$ [$_{\textsc{TP1}}$ \dots [$_{\textsc{vP}}$ \dots [$_{\textsc{TP2}}$ \dots [$_{\textsc{vP}}$ wh$_i$ ]]]]]\hspace{\fill}
   \b.\label{wh.240b}partial wh-movement $=$ \ref{wh.190b}\\{}[$_{\textsc{TyP}}$ Op$_i$ [$_{\textsc{TP1}}$ \dots [$_{\textsc{vP}}$ \dots [$_{\textsc{CP}}$ wh$_i$ \dots [$_{\textsc{TP2}}$ \dots [$_{\textsc{vP}}$ $t_i$ ]]]]]]\hspace{\fill}
   \b.\label{wh.240c}wh-fronting $=$ \ref{wh.190c}\\{}[$_{\textsc{TyP}}$ Op$_i$ [$_{\textsc{CP}}$ wh$_i$ \dots [$_{\textsc{TP1}}$ \dots [$_{\textsc{vP}}$ \dots [$_{\textsc{TP2}}$ \dots [$_{\textsc{vP}}$ $t_i$ ]]]]]]\hspace{\fill}\hspace{\fill}

We will not adopt this analysis; rather we claim that optional wh-movement of this sort is an exemplar of SSO. The analysis that we will propose in section 3 will be expressed in terms of a syntactic unity among all three constructions with wh-movement to the highest [Spec,CP] in all cases for wh-scope reasons. What differs is where the wh-phrase is spelled-out, which can only be links in the wh-movement chain. Typologically, languages differ in terms of what is allowed to spell-out where, while some languages allow several options, hence SSO.

The third type of construction we will consider is partial wh-movement with scope marking. Partial wh-movement is similar to what we find in Babine-Witsuwit'en, specifically when the wh-phrase moves to, and appears to stop at an intermediate [Spec,CP] with the corresponding surfacing of a marker at the scope marking position (usually matrix). This scope marker can take the form of a wh-word or a particle. \cite{fanselow:2006} provides a very detailed overview of the cross-linguistic profile of partial wh-movement shows the following examples from German and Albanian (citing \citealp{turano:1995}):

\ex.\label{wh.250}\ag.\label{wh.250a}was glaubst du wen$_i$ Irina $t_i$ liebt?\\
   what believe you who-\textsc{acc} Irina {} loves\\
   \trans{`Who do you believe that Irina loves?'}
   \bg.\label{wh.250b}was glaubst du was er sagt wen$_i$ Irina $t_i$ liebt?\\
   what believe you what he says who Irina loves\\
   \trans{`Who do you believe that he says that Irina loves?'}

\ex.\label{wh.260}\ag.\label{wh.260a}a mendon se Maria thot\"e se \c{c}far\"e ka sjell\"e burri?\\
   Q think that Mary says that what brought her husband\\
   \trans{`What do you think that Maria says that her husband brought?'}
   \bg.\label{wh.260b}a mendon se \c{c}far\"e thot\"e Maria se ka sjell\"e burri?\\
   Q think that what says Mary that brought her husband\\
   \trans{`What do you think that Maria says that her husband brought?'}

In German \ref{wh.250}, we see that \textit{was} `what' is used to mark scope, while the wh-word moves to the periphery of only the embedded clause. In Albanian \ref{wh.260}, we see the question particle \textit{a} marking matrix scope, while the wh-phrase \textit{\c{c}far\"e} `what' stays low. We will not go through the specifics of each language (although see \citealp{fanselow-cavar:2000} for a detailed account of German), but the main point is that these constructions are amenable to a SSO account of wh-constructions. Specifically, we claim that that the scope marker can be viewed as the instantiation of the Q-particle, while the lower wh-phrase corresponds to the argumental wh-phrase. If we merge this line of thinking with the notion that a question particle forms a constituent with the wh-phrase that it associates with, then a theory of SSO extends to not only the spell-out position of the wh-phrase, but also that of the Q-particle. Given that we know that wh-movement occurs in a cyclic fashion, \cite{fanselow:2006} observes that in German, the argumental wh-phrase can be partially moved to any intermediate peripheral position with copies of the scope marker appearing along the path of cyclic movement:

\ex.\label{wh.270}\ag.\label{wh.270a}\textbf{wen} denkst du dass sie glaubt dass Fritz meint dass sie liebt?\\
   who think you that she believes that Fritz means that she loves\\
   \trans{`Who do you think that she believes that Fritz means that she loves?'}
   \b.\textbf{was} denkst du \textbf{wen} sie glaubt dass Fritz meint dass sie liebt?
   \b.\textbf{was} denkst du \textbf{was} sie glaubt \textbf{wen} fritz meint dass sie liebt?
   \b.\textbf{was} denkst du \textbf{was} sie glaubt \textbf{was} fritz meint \textbf{wen} sie liebt?

The German data is somewhat similar to what we observed in Coptic. In Coptic, recall that relative tense marking appears when the lower copy of wh-phrase is spelled out, marking the position of where the higher copy would be otherwise pronounced. Likewise in German, \textit{was} is spelled-out in the positions that track higher copies of wh-movement. Fanselow correctly notes that it is slightly inaccurate to describe \textit{was} as simply a scope marker, since multiple copies appear -- it is more accurate to describe the \textit{highest} copy of movement chain as marking the scope of the clause. In this sense, German is different than Coptic, which only allows the relative tense marker to appear in the highest clause over which the wh-phrase takes scope. What this means is that the scope of the wh-phrase must be marked, languages differ in terms of the way they realise scope and the extent of the overtness of such marking: Coptic realises this through agreement in the scope taking clause; German allows the tracking of wh-movement and the spelling-out of multiple ``breadcrumb'' copies along the movement path to the scope-taking position; Babine-Witsuwit-en requires no overt scope marking at all.



%By varying the positions of where Q and the wh-phrase are spelled-out, along with whether or not Q is phonetically overt, we can unify the constructions that we have seen in this section: we can express a typology that describes the positions of both the wh-phrase and Q-particle we observed in \ref{wh.50}--\ref{wh.100}; we can account for the optional wh-movement case in Babine-Witsuwit'en, and we can describe the partial wh-movement and scope-marking constructions that we have just observed. We turn to this presently.

\section{SSO and the interaction of C, Q, and wh}
\subsection{Theoretical preliminaries}
In section 2, we asserted that the only requirement for the clause typing of an interrogative is the presence of an interrogative C. The effect that C has at the interfaces is basically that of a set-former: it takes a proposition and turns it into a set of propositions, which corresponds to the denotation of a question. One of the obvious functions of wh-movement is to fix scope, at least in the languages that have wh-movement; interrogative C triggers the movement of the wh-phrase to [Spec,CP], which gives rise to surface configuration we see. However, overt wh-movement clearly cannot be a necessity for scope-fixing, since we observe that in wh-in-situ languages, or languages that permit optional or partial wh-movement, the wh-phrase is not located in a position at which it takes scope. Usually in these cases, there must be a separate mechanism by which scope is fixed, e.g. through the use of a scope marking element or the positing of an abstract typing head in the clausal periphery. But yet again, none of these strategies are strictly necessary at surface syntax, e.g. in wh-in-situ languages that have no other overt question marker.

Taken at face value, one might entertain the idea that languages are parameterised in terms of whether wh-movement is used for scope fixing or not; in the latter case a scope marker is present, which itself may or may not be overt. However, this cannot be right -- in terms of the surface syntax, there is nothing absolutely necessary in the licensing of a wh-question, except perhaps the use of an overt wh-word. Its position, or the presence of overt particles or scope markers are in a sense, purely incidental. While it is reasonable to assume that the realisations of the surface structure is a reflex of the underlying syntactic mechanisms, it is cannot be the case that the licensing of a wh-question necessitates a certain surface configuration.

The counterpoint to this argument is that semantically, the basic meaning of questions crosslinguistically is not divergent. A simple wh-question, regardless of its surface configuration, must converge on the same meaning. One way of interpreting this is that there is a strict mapping between form and function, whereby a language what overtly moves a wh-phrase to a scope taking position \textit{uses} overt wh-movement to fix scope. By contrast, a wh-in-situ language does not use overt wh-movement to fix scope may choose to either use an overt scope marker, or it may not. This amounts to a situation of free choice: to fix scope a language may choose option A (wh-movement), option B (scope marking), or option C (do nothing). To us, this is not a desirable state of affairs. It seems more sensible to say that the syntax and semantics are more or less universally convergent -- here are the operations that result in questionhood; here are the operations that fix wh-scope; and finally, here are the operations that result in the surface configurations that we see.

The analysis that we are proposing here is that universally, the operations that determine questionhood and wh-scope are the same crosslinguistically, and it is only within the domain of surface configurations that languages vary. If we can then explain the variation of surface structures while keeping constant the processes that determine questionhood and wh-scope, we take steps towards the unification of syntactic structures.\footnote{Of course, one might argue that unification not necessary (or desirable), which calls into question the legitimacy of the entire syntactic enterprise. This leads to a completely different line of argumentation, we think, one that we cannot discuss here because it stems from a completely different set of starting assumptions.} We further argue that because we have a copy theory of movement, the explanation of crosslinguistic surface variation basically comes for free.

As a starting point, let us assume, following \cite{cable:2007}, that the formation of a wh-question requires not just interrogative C and a wh-phrase but also a Q-particle. What varies crosslinguistically are the relative positions of the wh-phrase and Q, as well as the form that the Q-particle takes; it could be overt or null, or it could be a wh-word or an independent morpheme. The view that a wh-phrase is closely linked to a Q-particle (for ease of exposition, let us call this the QP-approach) is gaining traction in the recent literature, and such a view has been especially espoused by \cite{hagstrom:1998}, \cite{cable:2007} and \cite{slade:2011}, who develop a detailed and formal account of the syntactic and semantic properties of Q. The specifics of each approach vary, but they converge more or less along the lines of the theoretical treatment of Q as a (variable over) choice functions. A choice function is a function that picks a member from a set of elements, and if the wh-phrase is seen as a set of entities (people for `who', things for `what' etc.), then the combination of Q and a wh-phrase yields an individual, what is commonly known in the literature as a wh-indefinite. Roughly as follows:\footnote{While the basic fact that a choice function picks a member from set remains unchanged, specific analyses vary. \cite{cable:2007}, for example, treats Q itself as a variable over choice functions, which needs to be bound by existential closure to yield indefinites, so \ref{sso.10} would more accurately be $\exists f.f(wh)$, $f$ a choice function.}

\ex.\label{sso.10}\a.wh$' = \{a,b,c\}$
   \b.Q(wh) $= x: x \in \{a,b,c\}$

Empirically, Japanese provides a very clear example of this  \citep{hagstrom:1998}:\footnote{A reviewer asks why there is a difference in politeness marking in \ref{sso.20a} and \ref{sso.20b}. The reason is that there are several possible question particles in Japanese, notably \textit{ka} and \textit{no}. Only \textit{ka}, however, is used in the formation of wh-indefinites and when used this way, allows both the polite and non-polite form of the verb. In questions, however, \textit{ka} is generally used to mark indirect questions and is found at the periphery of the embedded clause, whereas \textit{no} is used in a sentence final position at the periphery of the matrix clause. The additional complication is that \textit{ka} can be used sentence finally in a monoclausal question if the polite form of the verb is used. Therefore the only way to keep the particle constant to illustrate the general point of Q-movement is to use different forms of the verb, in line with \cite{hagstrom:1998} and \cite{cable:2007}. See \cite{miyagawa:1987} for a more detailed treatment of \textit{ka} vs. \textit{no}.}

\ex.\ag.\label{sso.20a}John-ga nani-ka-o katta\\
  John-\textsc{nom} what-Q-\textsc{acc} bought\\
  \trans{`John bought something.'}
  \bg.\label{sso.20b}John-ga nani-o kaimasita ka\\
  John-\textsc{nom} what-\textsc{acc} bought.\textsc{polite} Q\\
  \trans{`What did John buy?'}

In \ref{sso.20a}, \textit{nani-ka} `what-Q' is interpreted as an indefinite `something'. Crucially, since \ref{sso.20a} is a declarative, not a question, there is no interrogative C. However, \ref{sso.20b} is interpreted as a question and therefore has interrogative C. Japanese is a wh-in-situ language, and scope marking must therefore be accomplished through the use of the particle. Now, suppose that we say that \ref{sso.20b} is derived from \ref{sso.20a}, then it follows that the particle \textit{ka} has moved from a clause internal position to the periphery. In other words, in Japanese, C types a clause as interrogative, while wh-scope is fixed by moving the Q-particle to the periphery of the clause over which the wh-phrase takes scope. In a sense then, Japanese is similar to languages that have partial wh-movement with a scope marker, with the difference that in Japanese, the lowest copy of the wh-phrase is spelled out.

A reviewer points out an interesting set of facts observed by \cite{pesetsky:1987} that \cite{hagstrom:1998} uses as a diagnostic that \textit{ka} starts locally to the wh-phrase:

\exg.\label{sso.21}Mary-wa [John-ni nani-o ageta hito-ni] atta no?\\
   Mary-\textsc{top} John-\textsc{dat} what-\textsc{acc} gave man-\textsc{dat} met Q\\
   \trans{`Mary met the man who gave what to John?'}
   
In \ref{sso.21}, we observe that the wh-phrase can appear inside an island (complex NP). However, when \textit{nani} `what' is modified by \textit{ittai} `the hell', the result is ungrammatical:

\exg.*\label{sso.22}Mary-wa [John-ni ittai nani-o ageta hito-ni] atta no?\\
   Mary-\textsc{top} John-\textsc{dat} the.hell what-\textsc{acc} gave man-\textsc{dat} met Q\\
   \trans{`Mary met the man who gave what (the hell) to John?'}

Hagstrom proposes that \textit{ittai} marks the launching site of the question particle (Q) \textit{ka/no}, which explains the ungrammaticality of \ref{sso.22}. The generalisation is that \textit{ittai} may not appear inside a movement island because that would require Q to cross an island boundary. Crucially, when \textit{ittai} is outside the island, grammaticality is restored:

\exg.\label{sso.23}Mary-wa ittai [John-ni nani-o ageta hito-ni] atta no?\\
   Mary-\textsc{top} the.hell John-\textsc{dat} what-\textsc{acc} gave man-\textsc{dat} met Q\\
   \trans{`Mary met the man who gave what (the hell) to John?'}

Hagstrom considers this to be further evidence that \textit{ittai} does indeed mark the launching site of Q because in \ref{sso.23}, Q would not have to cross an island boundary to get to the periphery. The reviewer who raised this issue notes that this counts as counterevidence to an SSO theory of wh-movement because this suggests that Q moves independently of the wh-phrase, which is truly left in-situ, without any movement whatsoever, i.e. there are no intermediate copies.

While this observation is correct when interpreted in isolation, it presents only half of Hagstrom's analysis. Hagstrom is very careful to note that the edge of the island boundary marks the ``launching site'' of Q, not its base generation site. Ultimately, he settles on an analysis where the launching site of Q need not be identical to the base generation site of Q.\footnote{For interested readers, \cite{hagstrom:1998} tackles this issue in Chapter 4, addressing the question of remote vs. local generalisation of Q.}

So far, we have seen two ways of marking wh-scope: ``standard'' wh-movement to a scope taking position, and marking of scope through the use of a Q particle, with the wh-phrase moving to an intermediate position or staying in-situ. There is a third logical possibility, where there is no movement of both the wh-phrase and Q-particle. In this case, neither element is a viable candidate to mark scope and the only remaining possibility that is interrogative C itself is somehow implicated in scope marking. Consider Sinhala (\cite{slade:2011}, citing \cite{gair-sumangala:1991} and \cite{hagstrom:1998}):

\ex.\ag.\label{sso.30a}mokak d{\textschwa} w{\ae}tuna\\
   what Q fell-A\\
   \trans{`Something fell.'}
   \bg.\label{sso.30b}mokak d{\textschwa} w{\ae}tune?\\
   what Q fell-E\\
   \trans{`What fell?'}

\exg.\label{sso.31}Sunil mon{\textschwa}wa d{\textschwa} kieuwe?\\
   Sunil what Q read-E\\
   \trans{`What did Sunil read?'}

Like Japanese, the Q-particle \textit{d\textschwa} in Sinhala is merged in a position adjacent to the wh-phrase to yield an indefinite, as in \ref{sso.30a}. However, unlike Japanese, when forming a question, Sinhala does not resort to Q-movement to the periphery, which means that neither wh-movement nor Q-movement/marking can be responsible for fixing the scope of the wh-word. Instead, the verb ending \textit{-e} is responsible for marking the scope of the wh-phrase. \cite{slade:2011} analysis involves the observation that the \textit{-e} verbal marking in Sinhala is also responsible for focus, which then triggers head movement of the verb from V to I to Foc. If we follow \cite{rizzi:1997} in assuming that Focus is a head within the articulated CP layer, then this does not run counter to our claim that wh-questions are clause typed by interrogative C, whereas scope marking is divorced from clause typing, and languages differ in terms of the strategy they employ to do so.\footnote{Although this is something that we cannot address in detail here, a non-trivial follow-up to this issue is to what extent wh-movement is driven by focus. As far back as \cite{huang:1982}, it has been observed that wh-movement and focus movement interact in intricate ways; more modern analyses \citep{beck:2006} suggest that the set of propositional alternatives in wh-questions are focus alternatives. Empirically, the facts are rather complex. For example, in Babine-Witsuwit'en, \cite{denham:2000} argues that wh-movement is not driven by focus, since there is a dedicated focus marker for overt elements that bear focus. For purposes of this paper, since we are primarily concerned with expressing a coherent system of interrogativity, with an emphasis on clause typing and scope marking, we will adopt the approach that wh-movement is driven by the need to satisfy the EPP on some C head, and intentionally blur the distinction between as to whether the C head is ``purely'' interrogative, e.g. Force, or interrogative by means of focus alternatives.}

Japanese and Sinhala are traditionally known as wh-in-situ languages, and it is certainly not the case that Q-particles and QPs only appear in wh-in-situ languages. Cable's (\citeyear{cable:2007}) work on Tlingit shows clear empirical evidence that even in wh-movement languages, QPs and QP-movement can coexist alongside QPs giving rise a wh-indefinite meaning. Tlingit is a Na-Dene language, a phylum to which the Athabaskan languages also belong. However, unlike Babine-Witsuwit'en, which is an Athabaskan language, it allows free word order variation in declarative sentences -- there appears to be a rough preference of SOV, but any order is possible. Crucially, in wh-questions, the possible word orders are substantially restricted, such that the wh-phrase must always precede the predicate. Data from \cite[63--66]{cable:2007}:

\ex.\ag.aad\'ooch s\'a k\textul{g}wat\'oow y\'a x'\'ux'?\\
   who.\textsc{erg} Q he.will.read.it this book\\
   \trans{`Who will read this book?'}
   \bg.aad\'ooch s\'a y\'a x'\'ux' k\textul{g}wat\'oow?\\
   who.\textsc{erg} Q this book he.will.read.it\\
   \bg.y\'a x'\'ux' aad\'ooch s\'a k\textul{g}wat\'oow?\\
   this book  who.\textsc{erg} Q he.will.read.it\\
   \bg.*\label{sso.40d}y\'a x'\'ux' ak\textul{g}wat\'oow aad\'ooch s\'a?\\
   this book he.will.read.it who.\textsc{erg} Q\\

As can be seen from the paradigm above, \ref{sso.40d} is disallowed when interpreted as a question. However, QPs can follow predicates if they are interpreted as a wh-indefinite:

\exg.\label{sso.50}y\'a x'\'ux' akw\textul{g}wat\'oow aad\'ooch s\'a\\
   this book he.will.read.it who.\textsc{erg} Q\\
   \trans{`People will read this book.'}

\exg.\label{sso.60}k\'eet a\textul{x}\'a daa s\'a\\
   killer.whale he.eats.it what Q\\
   \trans{`A killerwhale will eat anything.'}

Cable concludes that QP-fronting to a position preceding the predicate is obligatory and therefore, Tlingit must count as a wh-movement language. To summarise, then, we now have the following ingredients in our syntactic framework for questions: 1) an interrogative C, which must be universally present in questions for clause typing purposes by forming a set of propositions; 2) a Q-particle, whose job is to combine with the wh-phrase to form an existential indefinite; 3) a wh-scope mechanism, which varies from language to language. In what follows, we will focus on point 3 by expressing a number of parameters that allows us to capture the crosslinguistic variation we observed in the previous above.

\subsection{Crosslinguistic variation of SSO}
In order to express the parameters we need to capture the variation we observe, we first need to establish the dimensions of variation we need. We start out with a basic schematic for wh-questions:

\ex.\label{sso.70}\begin{forest} baseline
      [CP, for tree={parent anchor=south, child anchor=north, align=center, base=top, l sep=1em, s sep=1em}
      [$\langle$wh Q$\rangle$] [C$'$
      [C$_Q$] [TP
      [\ldots] [\ldots \\ QP
      [wh] [Q]
      ]]]]
    \end{forest}

The principal claim here is that in the configuration above, QP always moves to the periphery, and what varies cross-linguistically is how much of the QP is spelled out and where. Since spelling out is a syntax--PF interface effect, it stands to reason that there must be some information that is encoded in syntax that is interpretable at the PF-interface that basically states ``spell-out this copy''. Given the system that we are developing here, it is necessary for us to reject a universal rule of ``spell-out highest copy''. If so, then there are four logical possibilities:

\ex.\label{sso.80}\a.\label{sso.80a}spell-out QP at the periphery
    \b.\label{sso.80b}spell-out QP in-situ
    \b.\label{sso.80c}spell-out Q in the periphery and wh in-situ
    \b.\label{sso.80d}spell-out wh in the periphery and Q in-situ

Note that \ref{sso.80} is expressed in terms of spelling out at the periphery, rather than simply in terms of movement. In general, there appears to be a preference for wh-elements, if they move, to be spelled-out at the left periphery. Recall the crosslinguistic distribution shown in \ref{wh.50}--\ref{wh.100}. We observe the following possibilities in terms of the relative positions of the wh-phrase and particle:

\ex.\label{sso.90}\a.\label{sso.90a}Q wh \dots  (Initial Q wh; Hopi)
   \b.\label{sso.90b}wh Q \dots  (Initial wh Q; Tlingit)
   \b.\label{sso.90c}wh \dots\ Q (Initial wh, final Q; Vata)
   \b.\label{sso.90d}Q \dots wh (Initial Q, wh-in-situ; Tumbuka)
   \b.\label{sso.90e}\dots\ wh Q \dots (In-situ wh Q; Sinhala)
   \b.\label{sso.90f}\dots\ wh\dots\ \ Q (In-situ wh, final Q; Japanese)

Let us consider each option in turn. If we take into account the headedness of the QP, then a head initial QP that is spelled out in the left periphery will describe initial Q-wh configurations like Hopi as in \ref{sso.90a}. If the QP is head final and is spelled out at the left periphery, then we essentially arrive at Cable's \citeyear{cable:2007} analysis of Tlingit as in \ref{sso.90b}. We briefly postpone discussion of \ref{sso.90c}, which involves a discontinuous structure. Initial Q, wh-in-situ languages like Tumbuka simply involve QP movement, followed by spelling out of Q in the periphery and wh in the lower position as in \ref{sso.90d}. QP-in-situ structures like \ref{sso.90e} are found in Sinhala, and can be explained by spelling out of the entire lower QP copy. As mentioned above, Sinhala then resorts to verb movement and corresponding verbal morphology to mark scope; we will return to this. Finally, \ref{sso.90f} corresponds to a very typical wh-in-situ language with a question particle, such as Japanese, which involves spelling out of the lower copy of wh but a higher copy of Q in the right periphery. The issue as to why languages choose to spell elements in the left versus right periphery is not something we will tackle in this paper, although interested readers are pointed to \cite{hagstrom:1998}, who devotes a substantial part of his discussion to the phenomenon of ``Q-migration'', which precisely involves movement of Q from a clause internal to a clause peripheral position.

Discontinuous structures like \ref{sso.90c}, exemplified by Vata, are slightly more complicated, but there is nothing inherent in our system that prevents spelling out the wh-phrase in the left periphery and followed by the abovementioned Q-migration of the particle to the right periphery. What is important is that scope marking is nevertheless achieved in such a configuration -- assuming that both wh and Q are at the peripheries, then what is relevant for us is that scope is marked appropriately.

We have emphasised throughout this paper that the single unwavering requirement of clause typing is the presence of interrogative C. There is also a general requirement to mark scope, which we argued was the reason why Sinhala, which has the QP completely in-situ, resorts to verb movement and morphology for scope marking. However, while there is a crosslinguistic preference to mark scope overtly, it is by no means a language universal. While it is probably universal (and to an extent theoretically necessary) that the scope of a wh-phrase must be fixed, it is not the case that scope marking is universally overt. In the next section, we discuss the issue of Quantifier Raising, which is some sense is the opposite of scope marking -- QR is precisely used to fix the scope of a quantifier that overtly is in the wrong place for interpretation. Returning to wh-constructions, many languages freely permit wh-in-situ constructions with no corresponding particle, such as Babine-Witsuwit'en, discussed above. In order to account for clause typing, \cite{denham:1997,denham:2000} posited the existence of a Ty(ping)P, which hosts an operator that binds a wh-phrase. Wh-movement, in turn, is determined by the (optional) projection of C. The tree below shows a representive structure of a partial wh-movement case, which is a modified version of Denham's analysis updated to be more in-line with modern structures:

\ex.\label{sso.100}\begin{forest} baseline
      [TyP, for tree={parent anchor=south, child anchor=north, align=center, base=top, l sep=1em, s sep=1em}
      [Op$_k$] [Ty$'$
      [Ty] [TP
      [T] [\dots \\ VP
      [V] [CP
      [wh$_k$] [C$'$
      [C] [TP
      [T] [\dots \\ VP
      [V] [\dots \\ $t_k$]
      ]]]]]]]]
    \end{forest}

Essentially, Denham's analysis is a hybrid of wh-movement coupled with unselective binding. However, under current theoretical assumptions, an approach that allows the optional projection of C is untenable, specifically in the context of phase theory and feature inheritance of T from C \citep{chomsky:2008}. Under Denham's system, the projection of C entails the movement of wh. SSO provides a simple answer to this problem:

\ex.\label{sso.110}\begin{forest} baseline
      [CP, for tree={parent anchor=south, child anchor=north, align=center, base=top, l sep=1em, s sep=1em}
      [$\langle$Q wh1$\rangle$] [C$'$
      [C$_Q$] [TP
      [T] [\dots \\ VP
      [V] [CP
      [$\langle$\textst{Q} wh2$\rangle$] [C$'$
      [C] [TP
      [T] [\dots \\ VP
      [V] [\dots \\ $\langle$\textst{Q} wh3$\rangle$]
      ]]]]]]]]
    \end{forest}

Under our approach, the analysis for optional wh-movement, or any partial wh-movement construction would be as follows: assuming that the angled brackets denote copies, then QP must move to the highest [Spec,CP] to fix matrix scope. The wh-phrase can be spelled out at any of the positions labelled wh1, wh2, or wh3. Recall that languages differ in terms of the overtness of Q, so in a language like Babine-Witsuwit'en, it would simply be the case that Q is silent, since Denham does not note the existence of a question particle. In a sense, this is similar in spirit to Denham's analysis, without the need for positing the optionality of C or introducing unselective binding into a wh-movement structure. In the languages that do permit partial wh-movement with overt scope marking, then the natural explanation would be that in these languages, we have spell out of the wh-phrase in the wh2 position, followed by the spell out of Q in the matrix position.

A consequence of our analysis is that in a language like Sinhala, where the entire QP is in-situ, we must posit that the QP also moves for scope reasons, but only the entire lower copy is spelled out. Although not universally necessary, Sinhala chooses to mark scope by verb movement to C:

\ex.\label{sso.120}\begin{forest} baseline
      [CP, for tree={parent anchor=south, child anchor=north, align=center, base=top, l sep=1em, s sep=1em}
      [$\langle$\textst{wh Q}$\rangle$] [C$'$
      [TP [subj] [T$'$
      [VP [$t_{subj}$] [V$'$
      [$\langle$wh Q$\rangle$] [$t_V$]]]
         [$t_V$]]] [V+T+C]]]
    \end{forest}

Before we conclude this section, there is one final issue that we have not yet mentioned. A careful reader would have noticed that the logical possibilities of spell-out positions do not correspond to what we find empirically. These are repeated below:

\ex.\label{sso.130}\a.\label{sso.130a}spell-out QP at the periphery
    \b.\label{sso.130b}spell-out QP in-situ
    \b.\label{sso.130c}spell-out Q in the periphery and wh in-situ
    \b.\label{sso.130d}spell-out wh in the periphery and Q in-situ

\ex.\label{sso.140}\a.\label{sso.140a}Q wh \dots  (Initial Q wh; Hopi)
   \b.\label{sso.140b}wh Q \dots  (Initial wh Q; Tlingit)
   \b.\label{sso.140c}wh \dots\ Q (Initial wh, final Q; Vata)
   \b.\label{sso.140d}Q \dots wh (Initial Q, wh-in-situ; Tumbuka)
   \b.\label{sso.140e}\dots\ wh Q \dots (In-situ wh Q; Sinhala)
   \b.\label{sso.140f}\dots\ wh\dots\ \ Q (In-situ wh, final Q; Japanese)

All of \ref{sso.140} can be explained by some version of \ref{sso.130a}--\ref{sso.130c} if we take into account the left vs. right periphery and the headedness of the QP. However, as far as we know, there is no language that instantiates \ref{sso.130d} -- that is, there is no language that spells out a wh-phrase in the periphery while spelling out Q in its base position.

The main reason for this is inherent in the QP-approach. In wh-questions, the reason why QP-fronting is even possible is because the syntactic relationship that is involved is one that holds between C and Q, rather than C and wh, as is the traditional assumption. This means that if the EPP on C is to be satisfied by Q, then the QP should move. This immediately raises the question as to how we can arrive at discontinuous structures, whereby only Q seems to move but the wh-phrase stays low. In our terms, this involves the spelling out of Q at the periphery but wh-in-situ, which is commonly found in language. We mentioned above Hagstrom's (\citeyear{hagstrom:1998}) notion of Q-migration, which involves the movement of Q from a clause-internal position to a peripheral one. \cite{cable:2007} develops this further by positing that wh-in-situ languages are typologically split between ``Q-projection'' languages like Sinhala, in which Q takes the wh-phrase as its complement, versus ``Q-adjunction'' languages, where Q adjoins to the wh-phrase, offering Q more mobility when triggered to move by interrogative C, giving rise to the movement of Q from to the periphery in a language like Japanese, shown schematically as follows:

\ex.\label{sso.150}\begin{forest} baseline
      [CP, for tree={parent anchor=south, child anchor=north, align=center, base=top}
      [CP [TP [subj] [T$'$
      [VP [$t_{subj}$] [V$'$
      [whP [whP] [\textst{Q}]] [V]]]
         [T]]] [C]] [Q]]
    \end{forest}

A related tentative (though indirect) solution is also found in Chomsky's (\citeyear{chomsky:2013,chomsky:2015}) recent work on labelling. Under Chomsky's new system of labelling, Merge comes for free and is not feature-driven in the sense of earlier versions of Minimalism. The primary driving force in the interpretation of syntactic objects is the notion of labelling. Correspondingly, the inability to label some syntactic object gives rise to the need for displacement. \cite{chomsky:2013} provides the following example: if merge applies to \{H, XP\}, H a head and XP a non-head, then the labelling algorithm will select H as label, the usual case. So, in the case of Q-complementation, then if merge applies to \{Q, whP\}, then resulting syntactic object must be labelled Q(=QP). One of the interesting consequences of Chomsky's system is that is merge applies to two items, both non-heads, e.g. \{XP, YP\}, then the resultant object is unlabellable, causing it to crash when sent to the interface. In order to remedy this, either XP or YP must move (internal merge) and the labelling algorithm then selects the residue as label. For example, if XP moves, then Y is the label of the \{XP, YP\}, XP a (lower) copy. There is, however, one exception to the rule of labelling \{XP, YP\}: if XP and YP share relevant features such as \{$\phi$, $\phi$\} in the case of \{DP-subject,TP\} or more relevant for us, \{Q, Q\} in the case of questions. The upshot of this is that terminal point of the derivation of a question is to allow the resultant syntactic object to be labelled \{Q, Q\}.

Chomsky does not discuss this in any detail, but since labels can only be drawn from the pair of objects that undergo merge, this means that a wh-in-situ construction must involve a Q-bearing element to merge with interrogative C$_Q$. Assuming that C$_Q$ contributes one of the Q-features in a \{Q, Q\} sharing structure, the other must come from somewhere. So if one is serious about adopting such a theory of labelling, one must either 1) merge a unselective binder with a Q feature with C$_Q$, or 2) internal merge an existing QP with C$_Q$ and pronounce a lower copy to yield a wh-in-situ construction. In the case of Q-projection languages, the internal merging of Q must necessarily be the object QP(=\{Q, whP\}). This essentially boils down to QP-movement that we have already discussed for Sinhala and Tlingit.

In the case of a Q-adjunction language, which involves the adjunction of Q to whP. Crucially, since this is phrasal adjunction, Q cannot be a head that adjoins to a phrase, i.e. Q is a non-head. There are now two ways of considering this problem, both of which yields the desired result. Chomsky invokes pair merge in the case of adjunction (denoted by ordered pairs in angled brackets), which results in $\langle$whP, QP$\rangle$. For concreteness, let us assume that adjunction pair merged structures always result in the host as the first element of the pair, and the adjunct as the second, e.g. $\langle$host, adjunct$\rangle$. Since PP-adjuncts are clearly possible at the vP level, it must be the case that $\langle$VP, PP$\rangle$ is labellable, unlike \{VP, PP\}. Presumably, it must be the host to which the adjunct attaches that projects -- the labelling algorithm can select the first member of the pair merged structure as label. In this case, a syntactic object with $\langle$whP, QP$\rangle$ will be labelled [wh], and the need to derive \{Q, Q\} can only target QP (not wh) for internal merge, yielding a Japanese type language.

The second approach is to assume that what \cite{cable:2007} calls a Q-adjunction language is not adjunction in the true sense, since Q is a variable over choice functions and it is necessary for the syntax and semantics to converge, unlike a true adjunct which, by definition, must be optional. Pursuing such a line of thinking the syntactic object would not be ordered pair $\langle$whP, QP$\rangle$ but rather the unordered set \{whP, QP\}. This syntactic object is unlabellable. Recall that the resolution to an unlabellable structure is either feature sharing or internal merge. If we assume that Q-adjunction structures do not involve feature sharing in the relevant sense,\footnote{In any case, saying that there is feature sharing between whP and QP, say of [wh] or [Q], which then projects, basically reduces the syntactic object into a contiguous string indistinguishable from QP, an undesirable result because what we aim to capture here is the detachability of wh and Q.} then the only way such an object can be labelled is through internal merge. Since we need the residue to be labelled [wh] (it is a wh-in-situ construction), and we need the root to be labelled \{Q, Q\}, the only possibility is that QP is forced to move, leaving whP behind.

To sum up this section, we have argued that in the presence of interrogative C, the QP must always move (internal merge) to the periphery for scope and labelling reasons.\footnote{A more general statement about this labelling approach might be to say that question scope is interpreted at the point at which the syntactic object is labelled \{Q, Q\}. This effectively allows us to derive scope as an epiphenomenon of the labelling algorithm. At this stage, it is unclear as to whether such an approach is tenable.} Consequently, the availability of wh-in-situ or QP-in-situ must necessarily involve the spell-out of lower copies. We now turn to the issue of QR, which we explore as a parallel to a wh-in-situ construction in the sense of the spelling out of quantifier's lower copy that is not a scope taking position.



\section{On Quantifier Raising}
Assuming the mechanism of SSO for scope marking in cases of wh scope the question that naturally arises is whether there is something special about wh questions -- perhaps linked to the fact that they introduce propositional alternatives and that this can only be achieved wth certain syntactic technology -- or, alternatively, whether the mechanisms described are general and apply to scope setting in a more general way.  The latter would obviously be the preferred option.  In this connection we need to turn, however briefly, to the question of QR.  It is clear, to begin with, that the comparison between the ingredients involved in the cases of wh and in the case of QR are not the same.  The main difference is that there is no equivalent of an interrogative C in scope assignment via QR.  Now, while a majority of scholars agrees that some rule with the effect of QR is required, taking this effect to be scope assignment and type mismatch repair, there is also widespread skepticism concerning whehter QR is the right rule.  The skepticism originates mostly in the realisation that the unrestricted nature of QR ought to produce more scope combinations than what is actually observed. This is what\cite{szabolcsi:1997c} calls the
\emph{semantically blind rule of scope assignment}.  She writes:

\ex.  [this rule] \ldots roughly speaking ``prefixes'' an expression
$\alpha$ to a domain $\mathcal{D}$ and thereby assigns scope to it
over $\mathcal{D}$, irrespective of what $\alpha$ means and
irrespective of what operator $\beta$ may occur in $\mathcal{D}$:
\ex.[1] The semantically blind rule of scope assignment\\
 $\alpha[_{\mathcal{D}} \ldots \beta \ldots] \Rightarrow \
\alpha$ scopes over $ \beta$
\begin{flushright}
\cite[109]{szabolcsi:1997c}
\end{flushright}

Furthermore, save for a few relatively controversial cases where QR is argued to be overt \citep{fox-nissenbaum:99} QR is a covert operation.  In our terms, it would appear that contrary to wh elements which -in wh movement languages- show a preference for highest-copy spellout, in cases of QR, there is a preference for  \textit{lower copy spellout}.  QR and wh-scope are not the mirror image of one another, of course, since other movement may have taken place before QR.  To summarise, QR seems to have the following properties:


\ex. \label{qr-props}
\a. As a rule of grammar, QR targets a specific type of element(s) (quantificational elements) \label{target}
\b.It targets a single position (adjoined to IP) \label{position}
\c. It has \textit{dedicated} locality conditions (clause-boundedness). \label{locality}
\d. It always operates covertly. \label{covertness}
\e.  It is semantically blind


Empirically, it is well known that not all different possible scopes are derivable and more importantly there so-called \textit{surface scope} languages such as Japanese or Korean where as established as early as \citet{kuno:1973} scope ambiguities only arise if one scope bearing element has overtly moved over another.  Thi sis clear evidence in favour of taking scope assignement in general to be regulated by SSO, deriving reconstruction, though in many cases there will be interfering factors that obscure the effects, e.g. scrambling.
Within the theory developed in this paper we can sketch an approach to QR along the following lines.  We adopt the feature-based approach to scope assignemt due to \citet{beghelli-stowell:1997a} as an answer to the issue of overgeneration.  The structure \citet{beghelli-stowell:1997a} propose is given in \Next:

\ex.

\begin{forest}
[RefP, for tree={parent anchor=south, child anchor=north, align=center, base=top} [G(group)QP]
[CP[ WhQP]
[AgrS-P [C(ounting)QP]
[DistP [D(ist)QP]
[ShareP [GQP]
[NegP [N(eg)QP]
[AgrO-P [CQP][VP]
]]]]]]]
\end{forest}

Abstracting away from the presence of Agr nodes, this structure has dedicated scope positions for different types of QPs.  Scope for a particular DP is the result of either movement of the QP to the spec of the releevant scope head, or, we can extend the proposal, the result of the establishement of an AGREE relation between the scope head and the QP.  The general approach runs as fllows:  Suppose that the scope-marking functional heads optionally project, unless forced to do so.  Suppose further that subjects always move out of the vP phase to Spec,TP, for independent reasons.

Thus, for some object QP, if DistP does not project, QP-obj must remain in the vP -given the  PIC- and has, as a result narrow scope or rather it has no scope to speak of. Suppose now that DistP does project, then QP-obj must move to the vP-edge in order to be able to enter into an AGREE relation with Dist. The QP-obj is then interpreted distributively with scope corresponding to the Dist head's position.
Take now the case where one of the scope heads, say Dist, is overt, then the QP must move to the vP-edge, since there is no longer the choice of optionally projecting DistP and the reading will always be a distributive one.  For reasons that go beyond the reach of this paper, `every' forces Dist to project, while `all' allows Dist to optionally project.
To complete the picture, for non-universals, one option would be to  suppose that there's some projection, say ExistP above IP, where existential closure usually applies \citep{kratzer:2005}. So in the case of  `Everyone loves someone', `everyone' must always raise, since it's the subject. Then depending on whether or not ExistP is projected,  wide/narrow existential scope can be derived.

Sketchy though it is, the above account suggests that it is possible to unify the scope assignement mechanisms that are seen in detail in the case of wh scope with the way scope is assigned in general.  Setting aside, for good reason, the role played by interrogative C, it appears that scope in general involves two parts.  A scope marker/head equivalent to the Q particle and a QP equivalent to the wh word.  A case where these are overt is that of floating quantifiers.  As noted by \citet{dowty-brodie:1984} floating quantifiers fix the scope of the DP that they associate with at the position of the floating quantifier.  \citet{tsoulas:2003} proposed that they are indeed scope markers generated directly in the relevant scope heads.  But we can combine this insight with the more traditional stranding analysis in terms of the analysis of wh-scope.  In other words the [Q+DP] constituent will be exactly parallel to the [Q+wh] constituent that we encountered earlier.


\section{Conclusions}
In this paper we have claimed that a careful consideration of the mechanisms of scope assignment to wh elements in questions shows that the best analysis is in terms of selective spellout rather than other mechanisms.  When undersood properly, the same mechanism extends naturally to scope assignemnt in general given relatively netural assumptions about the phrase structure of scope and the relations between scope markers and scope takers.  If this analysis is on the right track, and there is no doubt a great deal remains to be done, scope mechanisms can be dispensed altogether as scope assignment can be achieved through other generally available means.


\newpage
%\begin{itemize}
%	\item TYPE 1: ALL CYCLIC Cs APPEAR
%	\item the appearance of a special complementiser and corresponding agreement on that surfaces on the predicate V or A
%    \item case marking on V or A, triggered by wh-movement
%    \item null operator movement in relative clauses
%    \item[]
%    \item TYPE 2: ONLY HIGHEST C APPEARS
%    \item operator-C agreement
%    \item depending on the nature of moved element, different Cs are used
%\end{itemize}
%
%Their analysis
%\begin{itemize}
%	\item movement is parasitic on Agree and
%    \item TYPE 1 is a morphological reflex of Agree
%    \item CP is a strong phase; all Cs are featurally identical
%    \item Activity condition is assumed (though unspecified in their paper)
%    \item[]
%    \item TYPE 2 involves a different interrogative C, where only the highest one bears Q
%\end{itemize}

% Other relevant issues regarding SSO:
% \begin{itemize}
% 	\item overt wh-movement is subject to islands
%     \item this entails that SSO is also subject to islands
%     \item what to do in multiple wh-questions in English? Does the lower one move?
%     \item LF-movement is not subject to islands = unselective binding
%     \item This entails two types wh-in-situ: 1) SSO (wh inside islands NOT OK); 2) unselective binding (wh inside islands OK)
%     \item Unselective binding split into two types: 1) base generation of scope marker in scope taking position (German partial wh-movement); 2) local generation of scope marker followed by movement to scope taking position (Q-movement)
% \end{itemize}

% Notes on QR:
% \begin{itemize}
% 	\item QR can be recast as unselective binding by a scope operator
%     \item Multiple quantifiers and ambiguous scope can be expressed as Absorption on the scope marker, i.e. Q$_{\forall>\exists}$ vs. Q$_{\exists>\forall}$
%     \item Example of floating quantifiers as scope markers, floating all/each = generalised D-operator.
%     \item Determiner quantifiers must QR $\lambda P\lambda Q.\forall x[P(x) \to Q(x)]$ because they take the NP as its first argument, and need to move to scope over the sentence.
%     \item In our terms, determiner quantifiers must be bound by a scope marking operator, cf. Q binding wh-phrase (which is a DP), vs. Scope binding quantified DP -- there's a parallel to be made here.
%     \item Floating quantifiers are generalised D-operators and don't need scope marking since they mark scope where they are, $\lambda Q\lambda P.\forall x[P(x) \to Q(x)]$, i.e. they take the VP as its first argument, i.e. they are not determiners
% \end{itemize}

\bibliography{scope-glossa}


\end{document} 