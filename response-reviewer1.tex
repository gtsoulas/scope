%%% response-reviewer1.tex ---
%%
%% Filename: response-reviewer1.tex
%% Description:
%% Author: George Tsoulas
%% Maintainer:
%% Created: Fri Dec  2 10:58:03 2016 (+0000)
%% Version:
%% Package-Requires: ()
%% Last-Updated:
%%           By:
%%     Update #: 0
%% URL:
%% Doc URL:
%% Keywords:
%% Compatibility:
%%
%%%%%%%%%%%%%%%%%%%%%%%%%%%%%%%%%%%%%%%%%%%%%%%%%%%%%%%%%%%%%%%%%%%%%%
%%
%%% Commentary:
%% I broke up the pages so that the notes fit.  We will remove the breaks afterwards.
%%
%%
%%%%%%%%%%%%%%%%%%%%%%%%%%%%%%%%%%%%%%%%%%%%%%%%%%%%%%%%%%%%%%%%%%%%%%
%%
%%% Change Log:
%%
%%
%%%%%%%%%%%%%%%%%%%%%%%%%%%%%%%%%%%%%%%%%%%%%%%%%%%%%%%%%%%%%%%%%%%%%%
%%
%% This program is free software; you can redistribute it and/or
%% modify it under the terms of the GNU General Public License as
%% published by the Free Software Foundation; either version 3, or
%% (at your option) any later version.
%%
%% This program is distributed in the hope that it will be useful,
%% but WITHOUT ANY WARRANTY; without even the implied warranty of
%% MERCHANTABILITY or FITNESS FOR A PARTICULAR PURPOSE.  See the GNU
%% General Public License for more details.
%%
%% You should have received a copy of the GNU General Public License
%% along with this program; see the file COPYING.  If not, write to
%% the Free Software Foundation, Inc., 51 Franklin Street, Fifth
%% Floor, Boston, MA 02110-1301, USA.
%%
%%%%%%%%%%%%%%%%%%%%%%%%%%%%%%%%%%%%%%%%%%%%%%%%%%%%%%%%%%%%%%%%%%%%%%
%%
%%% Code:

\documentclass[11pt]{article}

\usepackage{linguex}
\usepackage{todonotes}

\begin{document}
\title{Response to first reviewer's comments}
\author{this is the review starting \textit{the paper argues\ldots}}
\date{December 2016}

\maketitle
This was by far the most detailed of reviews, and we thank the reviewer for such careful and detailed consideration. In what follows we will provide responses and clarifications to the more important points raised by the reviewer.  There have been many changes to the paper to reflect what we include here in summary form.  Overall we have dealt with the majority of the comments except in one area where giving the issues the consideration that they deserve would require much more of both time and space than we had.

To begin with, the reviewer asks about the novelty/difference of our proposal with respect to the Q-theory as currently understood (to the extent that there is a generally accepted understanding).  In the proposals developed in the paper we argue for an SSO approach to various copies of the QP which is different from current theories. While the proposals we present here appear to be quite similar to what Cable (2007) proposes, internally, they are actually quite different. We did not see fit to include a full discussion of this issue in the paper because it is not our intention to launch a direct objection of Cable, but since the reviewer specifically raises the issue of Sinhala and Japanese, we will address it here.

Cable (2007:359) proposes the Q-movement parameter: Overt vs. Covert which he needs to explain the facts about Sinhala, which he claims is a Q-projection language, i.e. Q takes wh as a complement. On the other hand, Japanese is a Q-adjunction language, i.e. Q is adjoined to wh, which allows Q alone to move to the periphery. In this sense, what Cable wants to say is that Sinhala is a QP-in-situ language, i.e. an in-situ version of Tlingit, because it appears that Q is adjacent to the wh-phrase, Q appears at the edge of islands etc. However, if we look closely at the data that Slade (2011) presents (granted that Slade's thesis was a comprehensive study of Sinhala), we actually observe that Sinhala is more similar to Japanese (a Q-adjunction language) than it is to Tlingit. The data in Slade (2011:17--18) clearly show that Q can separate from the wh-phrase in certain contexts. The other issue is that there is a common assumption that the E-form of the verb in Sinhala makes scope, but we have argued in the paper that it cannot be so, because the E-form also appears in non-interrogative focus constructions. This means that an independent wh-scoping mechanism is still required.

The next point has to do with the issues surrounding Q-migration. The data that Hagstrom provides second point is quite compelling (and inconvenient to explain). Hagstrom goes to great lengths to explain that Q-migration is insensitive to islands and intervenors, but crucially that Q can only migrate to the island's edge. It cannot be a coincidence that in Sinhala Q is always wh-local unless there's an island, where it would then be found at the edge; in fact, Tlingit also does this, which is why Cable suggests that Tlingit and Sinhala are both Q-projection languages. But if we abstract away from the language specifics, and if Hagstrom is correct about Q-migration in Japanese, then Japanese, Sinhala and Tlingit are actually not too different. All of them have some kind of Q-migration that takes Q to the island's edge and in Sinhala/Tlingit, Q is spelled-out there, but in Japanese, Q is spelled out at the periphery. As a result, there no longer is a need for LF-movement that Cable assumes for Sinhala (as the reviewer points out as an objection). We take this to be a welcome result. Our proposals, of course, build on earlier work.

The other significant difference we propose is that we used SSO to account for discontinuous spell-out of different material in the two copies thus providing an analysis for discontinuous structures. While this has been claimed before (especially by Bo\v{s}kovi\'{c}) for other phenomena but our application here to QP constructions here is new. We have, however, essentially established and expressed coherently, we believe, the essential  underlying uniformity of syntax across the typological range of wh-constructions, without positing the need for Q-adjunction vs. Q-complementation, again a welcome simplification of the theory. This is not merely an issue of relegation of some phenomena to PF vs. LF. Whether something is applied at PF or LF is non-trivial -- there is no reason for PF to care about issues like uniformity of the copies at chain links; there are no crucial dependencies that hold between chain links at PF and it can spell-out anything anywhere (of course taking into account obvious PF constraints) to no detriment. LF, on the other hand, is quite different. Chain-link uniformity, which the copy theory entails, is by definition, problematic for LF, since the bulk of LF-movement operations are posited to establish operator--chain relations. It is easy to trivialise the issue and say that under a copy theory, LF-movement is simply recast in terms of the pronunciation of a lower copy, but this masks the real difficulty -- which is the question whether operator--variable chains that are formed by movement can be eliminated can be eliminated from the grammar, which is what the strongest version of the copy theory entails.

A further point made in the review specifically concerns Japanese. The point is while wh-phrases can freely appear inside movement islands, the inclusion of \textit{ittai} `the hell' to wh-phrases results in ungrammaticality. The point is that \textit{ittai} forces Q to move to the periphery from where it is (\textit{ittai} marks the launching site of Q), resulting in an island violation. However, as we show in the paper, this is only part of the picture because without \textit{ittai}, wh-phrases can freely appear in islands. This means that if we adopt an approach where Q is always merged locally to the wh-phrase, then it appears that Q can \textbf{sometimes} evade islands. Hagstrom's proposal to this is Q-migration (discussed above), which allows Q to escape the island. We present a relatively complete set of facts, and specifically an extra set of data that Hagstrom himself uses to argue for Q-migration, where Q is always merged locally to a wh-word before \textbf{overtly} moving, even out of an island if necessary. We argued in the paper that such overt movement (Q-migration or otherwise), cannot preclude the pied-piping of the wh-phrase with it and leaving the relevant copies. Discontinuous spell-out takes care of the rest.

Next, the reviewer raises an issue regarding  Japanese and Vata, she/he notes that if Q and wh- do not move together, then the semantics for the wh-word cannot be derived. As mentioned above, what we argue for is that the QP always moves together, what we have is discontinuous spellout. Given that Q is a variable over choice functions, multiple copies of [Q wh] will all be bound by either existential closure or, more relevantly, interrogative C. This results in the desired effect of having QP being interpreted as an wh-indefinite across all copies, while avoiding the issue of copy non-uniformity of operator--variable chains. Declarative C and existential closure will result in a sentence a single declarative proposition with a wh-indefinite, whereas interrogative C would further provide propositional set formation, i.e. a set of propositions formed from the possible alternatives of the wh-indefinite.\todo{Should we say, the desired wh-indefinite or question word...Just because there is the ``more relevantly...'' part. @George It is always a wh-indefinite. The difference is whether or not there's declarative C, which gives a proposition, or interrogative C, which gives a set of propositions.}

The final two points the reviewer raises concern multiple wh-questions, superiority effects and multiple wh-movement. These are good points but rather large issues in themselves and warrant a separate paper where the proposals can be further elaborated to account for these cases. As mentioned in the introduction of this response, for reasons of time and in order to keep the paper at a manageable length we have opted to set those issues aside for this work and return to them in future research. We have included an acknowledgement of this in the paper in footnote 14.



\end{document}


%%%%%%%%%%%%%%%%%%%%%%%%%%%%%%%%%%%%%%%%%%%%%%%%%%%%%%%%%%%%%%%%%%%%%%
%%% response-reviewer1.tex ends here
