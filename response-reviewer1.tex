%%% response-reviewer1.tex --- 
%% 
%% Filename: response-reviewer1.tex
%% Description: 
%% Author: George Tsoulas
%% Maintainer: 
%% Created: Fri Dec  2 10:58:03 2016 (+0000)
%% Version: 
%% Package-Requires: ()
%% Last-Updated: 
%%           By: 
%%     Update #: 0
%% URL: 
%% Doc URL: 
%% Keywords: 
%% Compatibility: 
%% 
%%%%%%%%%%%%%%%%%%%%%%%%%%%%%%%%%%%%%%%%%%%%%%%%%%%%%%%%%%%%%%%%%%%%%%
%% 
%%% Commentary: 
%% I broke up the pages so that the notes fit.  We will remove the breaks afterwards.
%% 
%% 
%%%%%%%%%%%%%%%%%%%%%%%%%%%%%%%%%%%%%%%%%%%%%%%%%%%%%%%%%%%%%%%%%%%%%%
%% 
%%% Change Log:
%% 
%% 
%%%%%%%%%%%%%%%%%%%%%%%%%%%%%%%%%%%%%%%%%%%%%%%%%%%%%%%%%%%%%%%%%%%%%%
%% 
%% This program is free software; you can redistribute it and/or
%% modify it under the terms of the GNU General Public License as
%% published by the Free Software Foundation; either version 3, or
%% (at your option) any later version.
%% 
%% This program is distributed in the hope that it will be useful,
%% but WITHOUT ANY WARRANTY; without even the implied warranty of
%% MERCHANTABILITY or FITNESS FOR A PARTICULAR PURPOSE.  See the GNU
%% General Public License for more details.
%% 
%% You should have received a copy of the GNU General Public License
%% along with this program; see the file COPYING.  If not, write to
%% the Free Software Foundation, Inc., 51 Franklin Street, Fifth
%% Floor, Boston, MA 02110-1301, USA.
%% 
%%%%%%%%%%%%%%%%%%%%%%%%%%%%%%%%%%%%%%%%%%%%%%%%%%%%%%%%%%%%%%%%%%%%%%
%% 
%%% Code:

\documentclass[11pt]{article}

\usepackage{linguex}
\usepackage{todonotes}

\begin{document}
\title{Response to first reviewer's comments}
\author{this is the review starting \textit{the paper argues\ldots}}
\date{December 2016}

\maketitle
This was by far the most detailed of reviews, and we thank the reviewer for such careful and detailed consideration. In what follows  we will provide responses and clarifications to the more important points raised by the reviewer.  There have been many changes to the paper to reflect what we include here in summary form.  Overall we have dealt with the majority of the comments except in one area where giving the issues the consideration that they deserve would require much more both time and space than we had.

To begin with, the reviewer asks about the novelty/difference of our proposal with respect to the Q-theory as currently understood (to the extent that there is a generally accepted understanding).  In the proposals developed in the paper we argue for an SSO approach to various copies of the QP which is different from current theories\todo{Norman:  is this correct?  if yes can we make a stronger statement?}. More significantly, we used this proposal to account for discontinuous spell-out of different material in the two copies thus providing an analysis for discontinuous structures\todo{Can we say that this is new and did not exist before?}. As a result, there no longer is a need for LF-movement that Cable assumes for Sinhala (as the review points out as an objection).  We take this to be a welcome result. Our proposals, of course, build on earleir work.  We have, however,  essentially established and expressed coherently, we believe, the essential  underlying uniformity of syntax across the typological range of wh-constructions, without positing the need for Q-adjunction vs. Q-complementation, again a welcome simplification of the theory.




\newpage




A further point made in the review concerns Japanese.   The point is that \textit{ittai} suggests that Q moves on its own to the exclusion of the wh-phrase.\todo{Norman, Can you add a sentence here to make this point more obvious to the editors too?  Also to make the connection with the immediately following sentence clearer} We present a relatively complete set of facts, and specifically an extra set of data that Hagstrom himself uses to argue for Q-migration, where Q is always merged locally to a wh-word before \textbf{overtly} moving, even out of an island if necessary. We argued in the paper that such overt movement (Q-migration or otherwise), cannot preclude the pied-piping of the wh-phrase with it and leaving the relevant copies. Discontinuous spell-out takes care of the rest.

Next, the reviewer raises an issue regarding  Japanese and Vata, she/he notes that if Q and wh- do not move together, then the semantics for the wh-word cannot be derived. As mentioned above, what we argue for is that the QP always moves together, what we have is discontinuous spellout. Given that Q is a variable over choice functions, multiple copies of [Q wh] will all be bound by either existential closure or, more relevantly, interrogative C. This results in the desired wh-indefinite reading in all copies, while avoiding the issue of copy non-uniformity of operator--variable chains. Interrogative C then provides propositional set formation, as usual.\todo{Should we say, the desired wh-indefinite or question word...Just because there is the ``more relevantly...'' part}




\newpage




The final two points the reviewer raises concern multiple wh-questions, superiority effects and multiple wh-movement. These are good points but  rather large issues in themselves and warrant a separate paper where the proposals can be further elaborated to account for these cases. As mentioned in the introduction, for reasons of time and in order to keep the paper at a manageable length we have opted to set those issues aside for this work and return to them in future research. We have included an acknowledgement of this in the paper on page XXXX or in footnote XXXX\todo{Norman I don't think that we have done this, but I think we should to keep him/her happy that we heard them.  What would be the best place to put them?}.



\end{document}


%%%%%%%%%%%%%%%%%%%%%%%%%%%%%%%%%%%%%%%%%%%%%%%%%%%%%%%%%%%%%%%%%%%%%%
%%% response-reviewer1.tex ends here
