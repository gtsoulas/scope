%%% response-reviewer1.tex --- 
%% 
%% Filename: response-reviewer1.tex
%% Description: 
%% Author: George Tsoulas
%% Maintainer: 
%% Created: Fri Dec  2 10:58:03 2016 (+0000)
%% Version: 
%% Package-Requires: ()
%% Last-Updated: 
%%           By: 
%%     Update #: 0
%% URL: 
%% Doc URL: 
%% Keywords: 
%% Compatibility: 
%% 
%%%%%%%%%%%%%%%%%%%%%%%%%%%%%%%%%%%%%%%%%%%%%%%%%%%%%%%%%%%%%%%%%%%%%%
%% 
%%% Commentary: 
%% 
%% 
%% 
%%%%%%%%%%%%%%%%%%%%%%%%%%%%%%%%%%%%%%%%%%%%%%%%%%%%%%%%%%%%%%%%%%%%%%
%% 
%%% Change Log:
%% 
%% 
%%%%%%%%%%%%%%%%%%%%%%%%%%%%%%%%%%%%%%%%%%%%%%%%%%%%%%%%%%%%%%%%%%%%%%
%% 
%% This program is free software; you can redistribute it and/or
%% modify it under the terms of the GNU General Public License as
%% published by the Free Software Foundation; either version 3, or
%% (at your option) any later version.
%% 
%% This program is distributed in the hope that it will be useful,
%% but WITHOUT ANY WARRANTY; without even the implied warranty of
%% MERCHANTABILITY or FITNESS FOR A PARTICULAR PURPOSE.  See the GNU
%% General Public License for more details.
%% 
%% You should have received a copy of the GNU General Public License
%% along with this program; see the file COPYING.  If not, write to
%% the Free Software Foundation, Inc., 51 Franklin Street, Fifth
%% Floor, Boston, MA 02110-1301, USA.
%% 
%%%%%%%%%%%%%%%%%%%%%%%%%%%%%%%%%%%%%%%%%%%%%%%%%%%%%%%%%%%%%%%%%%%%%%
%% 
%%% Code:

\documentclass[11pt]{article}

\usepackage{linguex}

\begin{document}
<<<<<<< HEAD
\title{Response to third reviewer's comments}
\author{this is the review starting \textit{this is a well written paper}}
\date{\today}

\maketitle


The reviewer's insightful and helpful comments have allowed us to clarify a number of issues in the paper and have led to several major modifications in the presentation and argument. We thus would like to begin by thanking the reviewer for the time and effort that (s)he put in commenting on the paper. We are pleased that we have been able to incorporate the vast majority of the comments.  More specifically, We have followed the reviewer's suggestions and incorporated earlier discussions of the idea of differential spell-out.  We have limited ourselves to the areas where this was most relevant to the concerns of the paper.

More importantly, the reviewer raises the issue of the connection that we make between the analysis of WH scope and the Beghelli and Stowell feature-based system for scope.  The reviewer suggests that it may be problematic, in the framework of the paper, that choice function  indefinites and Q-particles do not behave alike.  The reviewer recognises that this is not in itself a problem but goes on to point out that  Q-particles do not behave as existential quantifiers.  This is seen as problematic in the context of our suggestion (modelled on Kratzer's 2005) that there may be an existential position corresponding to $\exists$-closure.    Strictly speaking this is a correct observation as in the semantic literature it is claimed htat choice-function indefinites can also take lower scope.  However, the comparison with Q-particles in this respect is not really warranted because Q-particles are not taken to be existential quantifiers.  Although we have not taken an explicit stance on this question, the analysis of Q-particles due to Hagstrom and later Cable takes them to be variables over choice functions which can be bound either by existential closure or an interrogative C reulting in their differnt interpertation sin the languages that have wh-indefinites. Given that semantic analyses of closure specifically take it to be an unrestricted operation that can take place at different points (see most importantly Reinhart 1997, Winter 1997)  It is to be expected that, if this is so, $\exists$Ps may be found at different points.   If anything this behaviour supports our proposal since it underlines a parallel between choice function indefinites which can be ``bound'' at wide or narrow scope levels by closure operations (this is because the closure operation is not restricted) and SSO as applied ot QR.


The reviewer also asks why, under this account QR would be clause bound unlike Wh movement. This is, of course, a very good question and as we have pointed out there is no good account for this behaviour of QR in hte literature that assumes the classic version of the rule.   We have not incorporated an explicit discussion of this point in the paper but the assumption here is that the locality of QR issue is solved immediately under the Beghelli and Stowell system and our reinterpretation of its mechanisms in the following way: given that the effects of QR are subsumed under AGREE between a functional head and a DP it follows that QR will be restricted by the same locality constraints as AGREE.  A problem woud arise in a situation where there are no scope heads in a subordinate clause and there are in the matrix.  We know of no such cases and the reason why they woud be ruled out anyway is the independent principle of the Phase Impenetrability Condition.  In other words, by the time the higher scope heads have been merged the lower DPs would have been subject to TRANSFER.  
=======
\title{Response to first reviewer's comments}
\author{this is the review starting \textit{the paper argues}}
\date{\today}

\maketitle
 This was by far the most detailed of reviews, and we thank the reviewer for such careful consideration. The reviewer asks what the novel/difference of our proposal with respect to the Q-theory and we have done is to argue for an SSO approach to various copies of the QP and more significantly, discontinuous spell-out of different material in the two copies to account for discontinuous structures. This voids the need for LF-movement that Cable assumes for Sinhala (as the review points out as an objection). While it is true that our solution also resorts to interface effects (LF vs. PF), we have essentially established an underlying uniformity of syntax across the typological range of wh-constructions, without positing the need for Q-adjunction vs. Q-complementation. The review further points out the issue of Japanese, suggesting that \textit{ittai} suggests that Q moves on its own to the exclusion of the wh-phrase. We present a relatively complete set of facts, and specifically an extra set of data that Hagstrom himself uses to argue for Q-migration, where Q is always merged locally to a wh-word before \textbf{overtly} moving, even out of an island if necessary. We argue that such overt movement (Q-migration or otherwise), cannot preclude the pied-piping of the wh-phrase with it and leaving the relevant copies. Discontinuous spell-out takes care of the rest. The next point address Japanese and Vata, where the reviewer notes that if Q and wh do not move together, then the semantics for the wh-word cannot obtain. As mentioned above, we argue that the QP always moves together, what we have is discontinuous spellout. Given that Q is a variable over choice functions, multiple copies of [Q wh] will all be bound by either existential closure or more relevantly interrogative C. This results in the desired wh-indefinite reading in all copies, while avoiding the issue of copy non-uniformity of operator--variable chains. Interrogative C then provides propositional set formation, as usual. The final two points the reviewer raises has to do with multiple wh-questions, superiority effects and multiple wh-movement. These are rather large issues in themselves and warrant a separate paper. For reasons of time and space, it was simply not possible to include these facts at these time but we will be certain to turn our attention to these issues for future research.


>>>>>>> refs/remotes/origin/master

\end{document}


%%%%%%%%%%%%%%%%%%%%%%%%%%%%%%%%%%%%%%%%%%%%%%%%%%%%%%%%%%%%%%%%%%%%%%
%%% response-reviewer1.tex ends here
